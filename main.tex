%%%%%%%%%%%%%%%%%%%%%%%%%%%%%%%%%%%%%%%%%
% Их сургуулийн оюутны тезис  
% LaTeX Загвар
% Version 2.3 (25/3/16)
%
% Энэ загвар нь дараах сайтаас авсан загварын монгол хувилбар юм.
% http://www.LaTeXTemplates.com
%
% Version 2.x major modifications by:
% Vel (vel@latextemplates.com)
%
% Анхдагч загварын эх үүсвэр:
% Steve Gunn (http://users.ecs.soton.ac.uk/srg/softwaretools/document/templates/)
% Sunil Patel (http://www.sunilpatel.co.uk/thesis-template/)
%
% Загварын лиценз:
% CC BY-NC-SA 3.0 (http://creativecommons.org/licenses/by-nc-sa/3.0/)
%
%%%%%%%%%%%%%%%%%%%%%%%%%%%%%%%%%%%%%%%%%

%-------------------------------------------------------------------------------
%	PACKAGES AND OTHER DOCUMENT CONFIGURATIONS
%-------------------------------------------------------------------------------

\documentclass[
	11pt, % Баримтын фонтын хэмжээ, сонголт: 10pt, 11pt, 12pt
	oneside, % Хоёр талаар хэвлэж үдэхээр тохируулсан. Нэг тал бол комментыг арилга
	%chapterinoneline,% Нэг мөрөнд бүлгийн дугаар, нэрийг гаргах
	english, % babel багцын хэлний тохиргоо
	singlespacing, % Мөр хоорондын зай. Сонголтууд: singlespacing, onehalfspacing, doublespacing
	%draft, % Ноорог горимд шилжихийн тулд комментыг арилга(зураг, холбоос, hboxes гарахгүй)
	nolistspacing, % Хэрэв мөр хоорондын зай onehalfspacing эсвэл doublespacing бол, жагсаалтын мөр хоорондын зайг single болгохын тулд комментыг арилга
	%liststotoc, % Зураг/хүснэгт/бусад жагсаалтыг гарчигт оруулахын тулд комментыг арилга
	%toctotoc, % Uncomment to add the main table of contents to the table of contents
	%parskip, % Параграф хооронд зай оруулахын тулд комментыг арилга
	%nohyperref, % hyperref багцыг ачаалахгүй бол комментыг арилга
	headsepline, % Толгой мөрийн доогуур шугам татахын тулд комментыг арилга
]{MUST-Thesis} % Энэ класс файл нь баримтын бүтцийг тодорхойлно

\usepackage[utf8]{inputenc} % Олон улсын тэмдэгт оруулахад хэрэгтэй
\usepackage[T2A]{fontenc} % Олон улсын тэмдэгтийн гаралтын кодчилол
\usepackage[mongolian]{babel}
\usepackage{unicode-math}
\usepackage{minted}

\usepackage{titlesec}
\usepackage[table]{xcolor}
\usepackage{rotating} % эргүүлэх
\usepackage{multirow}

\usepackage{tcolorbox}
\usepackage{tabularx}
\usepackage{array}
\usepackage{colortbl}
\usepackage{dirtree}
\tcbuselibrary{skins}

\newcolumntype{Y}{>{\raggedleft\arraybackslash}X}
\newcolumntype{Z}{>{\centering\arraybackslash}X}

\tcbset{tab2/.style={enhanced,fonttitle=\bfseries,fontupper=\footnotesize\rmfamily,
		colback=white,colframe=black,colbacktitle=white,
		coltitle=black,center title,sharp corners=all}}

\usepackage{amsmath}

\usepackage{pgf}
\usepackage{tikz} % зураг зурах
\usetikzlibrary{shapes,arrows,automata}
\usepackage{subcaption}
\usepackage{xcolor}

\usepackage{pagecolor}
\definecolor{LigthBlue}{RGB}{102,255,255}
\definecolor{Beige}{RGB}{255,220,180}

\definecolor{OliveGreen}{cmyk}{0.64,0,0.95,0.40}
\definecolor{CadetBlue}{cmyk}{0.62,0.57,0.23,0}


\definecolor{lightlightgray}{gray}{0.9}
\usepackage{listings}
\renewcommand{\lstlistingname}{Эх код} % Програмын эх код хэвлэх
\lstset{
language={[x86masm]Assembler},			% Code langugage  [x86masm]Assembler C VHDL Phyton 
basicstyle=\ttfamily,                   % Code font
keywordstyle=\color{OliveGreen},        % Keywords font ('*' = uppercase)
commentstyle=\color{CadetBlue},         % Comments font
numbers=left,                           % Line nums position
numberstyle=\tiny,                      % Line-numbers fonts
stepnumber=1,                           % Step between two line-numbers
numbersep=5pt,                          % How far are line-numbers from code
backgroundcolor=\color{lightlightgray}, % Choose background color
frame=none,                             % A frame around the code
tabsize=2,                              % Default tab size
captionpos=t,                           % Caption-position = bottom
breaklines=true,                        % Automatic line breaking?
breakatwhitespace=false,                % Automatic breaks only at whitespace?
showspaces=false,                       % Dont make spaces visible
showtabs=false,                         % Dont make tabls visible
columns=flexible,                       % Column format
morekeywords={__global__, __device__},  % CUDA specific keywords
}

\usepackage[autostyle=false]{csquotes} % Ном зүйд хэлнээс хамаарсан хашилт оруулахад хэрэгтэй

\usepackage[backend=bibtex,natbib=true,sorting=none,sortcites]{biblatex} % Ном зүйд bibtex -г ашиглах

\addbibresource{references.bib} % Ном зүйн файл

\newcommand{\authorshipname}{Зохиогчийн эрх хамгаалал}
\newcommand{\abbrevname}{Товчилсон үгс}
\newcommand{\constantsname}{Физик тогтмолууд}
\newcommand{\symbolsname}{Таних тэмдэгтүүд}
\newcommand{\acknowledgementname}{Талархал}
\newcommand{\abstractname}{Хураангуй}
\newcommand{\conclusionname}{Дүгнэлт}
 
%-------------------------------------------------------------------------------
%	THESIS INFORMATION
%-------------------------------------------------------------------------------

\thesistitle{Прокси дахин шифрлэх схемийн туршилтын системийг хөгжүүлэх нь} % Таны ажлын нэр, нүүр болон хураангуй хуудсанд ашигласан. Өөр газарт бол \ttitle командыг хэрэглэнэ
\thesistitleeng{Developing Prototype System of Proxy Re-Encryption Scheme} % Таны ажлын нэр, нүүр болон хураангуй хуудсанд ашигласан. Өөр газарт бол \ttitleng командыг хэрэглэнэ
\thesistype{Бакалаврын төгсөлтийн ажил} % Удирдагчийн нэр, нүүр хуудсанд ашиглана. Дурын газарт бол \thesisname командыг хэрэглэнэ
\authorshort{А.Мягмарцэрэн} % Таны товч нэр, нүүр болон хураангуй хуудсанд ашигласан. Дурын газарт бол \shortname командыг хэрэглэнэ 
\authorlong{Амгаланбаатарын Мягмарцэрэн} % Таны бүтэн нэр, нүүр болон хураангуй хуудсанд ашигласан. Дурын газарт бол \longname командыг хэрэглэнэ 
\authorcode{B190970106} % Таны оюутны код, үзлэгийн хуудсанд ашигласан.  Дурын газарт бол \studentcode командыг хэрэглэнэ 
\addresses{b190970106@must.edu.com} % Таны хаяг, одоогоор ашиглаагүй. Өөр газарт бол \addressname командыг хэрэглэнэ
\phonenumber{99754252} % Таны хаяг, одоогоор ашиглаагүй. Өөр газарт бол \phonenum командыг хэрэглэнэ

\supervisor{доктор (Ph.D) В.Нямсүрэн} % Удирдагчийн нэр, нүүр хуудсанд ашиглана. Дурын газарт бол \supname командыг хэрэглэнэ
\reader{магистр Г.Баяр} % Шүүмжлэгчийн нэр, Дурын газарт бол \readname командыг хэрэглэнэ
\advisorA{доктор (Ph.D), Ц.Энхтөр} % Зөвлөгчийн нэр, Дурын газарт бол \advicenameA командыг хэрэглэнэ
\advisorB{магистр Ц.Манлайбаатар} % Зөвлөгчийн нэр, Дурын газарт бол \advicenameB командыг хэрэглэнэ
\degreeind{D061940} % Мэргэжлийн индекс, нүүр болон хураангуй хуудсанд ашигласан. Өөр газарт бол \degreeid командыг хэрэглэнэ
\degree{Мэдээллийн системийн аюулгүй байдал} % Боловсролын зэрэг, нүүр болон хураангуй хуудсанд ашигласан. Өөр газарт бол \degreename командыг хэрэглэнэ
\subject{Мэдээллийн сүлжээ, аюулгүй байдлын салбар} % Таны салбар, одоогоор ашиглаагүй. Дурынр газарт бол \subjectname командыг ашиглана
\keywords{мэдээллийн аюулгүй байдал, прокси дахин шифрлэлт} % Түлхүүр үгс, одоогоор ашиглаагүй. Дурын газарт бол \keywordnames командыг хэрэглэнэ
\department{Мэдээллийн сүлжээ, аюулгүй байдлын салбар} % Сургууль/тэнхмийн нэр, нүүр болон хураангуй хуудсанд ашигласан. Дурын газарт бол\deptname командыг хэрэглэнэ
\deptchair{доктор (Ph.D) Б.Мөнхбаяр} % Тэнхим/эрхлэгчийн нэр, нүүр болон хураангуй хуудсанд ашигласан. Дурын газарт бол\chairname командыг хэрэглэнэ
\group{Робот техникийн баг} % Судалгааны баг/тэнхмийн нэр, нүүр хуудсанд ашигласан. Дурын газарт бол \groupname командыг хэрэглэнэ
\faculty{\href{http://www.sict.edu.mn}{Мэдээлэл, Холбооны Технологийн Сургууль}} % Салбар сургууль/факультетийн нэр, нүүр болон хураангуй хуудсанд ашигласан. Дурын газарт бол \facname командыг ашиглана
\university{\href{http://www.must.edu.mn}{Шинжлэх Ухаан, Технологийн Их Сургууль}} % Их сургуулийн нэр ба веб хаяг. Дурын газарт бол \univname командыг хэрэглэнэ 

\hypersetup{pdftitle=\ttitle} % Pdf файлын гарчиг
\hypersetup{pdfauthor=\shortname} % Pdf файлын зохиогчийн нэр
\hypersetup{pdfkeywords=\keywordnames} % Pdf файлын түлхүүр үгс
\hypersetup{allcolors=black} % Pdf файлын бүх холбоос хар өнгөтэй

\begin{document}
\frontmatter % Агуулгын өмнөх хуудас дугаарлалт: i, ii, iii, iv... г.м.

\pagestyle{plain} % Тезисийн загварыг дуудах хүртэлх толгой мөрийн суурь загвар

%-------------------------------------------------------------------------------
%	TITLE PAGE
%-------------------------------------------------------------------------------

\begin{titlepage}
\pagecolor{LigthBlue} % нүүр хуудас өнгөтэй болгохыг хүсвэл comment-ийг авна.
\begin{center}

{\scshape\LARGE \univname\par} % Их сургуулийн нэр
{\scshape\Large \facname\par}\vspace{1 cm} % Их сургуулийн нэр

\begin{figure}[!htbp]
\centering
\includegraphics[scale=1.2]{figures/MUST_logo.jpg}
\end{figure}

\vspace{1cm}
\hfill \large{\longname} \\

\vspace{1.5cm}

{\huge \bfseries \ttitle\par}\vspace{0.4cm} % Тезисийн нэр

\vspace{3cm}
\textsc{\Large {\thesisname}}\\ % Тезисийн төрөл

\vfill

\large {Улаанбаатар хот} \\
 
\end{center}
\end{titlepage}

%-------------------------------------------------------------------------------
%	SUBTITLE PAGE
%-------------------------------------------------------------------------------

\begin{titlepage}
\begin{center}
\pagecolor{white}
{\scshape\LARGE \univname\par} % Их сургуулийн нэр
{\scshape\Large \facname\par}\vspace{0.5cm} % Их сургуулийн нэр

\vspace{2cm}
\hfill \large{\deptname} \\

\vspace{3cm}

{\huge \bfseries \ttitle\par}\vspace{0.4cm} % Тезисийн нэр

\vspace{2cm}

\begin{minipage}[t] {0.9\textwidth}
\begin{flushleft} 
\normalsize

Мэргэжлийн индекс: \degreeid \\
Мэргэжил: \degreename \\[2cm]

\begin{tabular}{l l}
\emph{Удирдагч:} & {\supname} \\% Удирдагчийн нэр
\emph{Зөвлөгч:} &{\advicenameA} \\ % Зөвлөгч нарын нэрс
& {\advicenameB} \\ % Зөвлөгч нарын нэрс
\emph{Гүйцэтгэгч:} &\hspace{1.4cm} {\shortname} \\ % Зохиогчийн нэр
\end{tabular}


\end{flushleft}
\end{minipage}

\vfill

\large {Улаанбаатар хот} \\
{\large 2022 он 6 сар}\\ % Date

\end{center}
\end{titlepage}

  % Нүүр хуудас
%-------------------------------------------------------------------------------
%	WORK PLAN & REVIEW PAGE
%-------------------------------------------------------------------------------

%---------------------ТӨЛӨВЛӨГӨӨНИЙ ХУУДСЫН ЭХЛЭЛ--------------------------------
\begin{titlepage}
\noindent Батлав. \deptname ын эрхлэгч: 
\begin{flushright}
\makebox[6cm]{\dotfill} /\chairname/ 
\end{flushright} 
Удирдагч:\makebox[6cm]{ \dotfill} /\supname/
%\begin{flushright}
%\makebox[4cm]{ \dotfill} /\supname/
%\end{flushright}
\begin{center}
\vspace*{0.5cm}
\textbf{{\large \textsc{Дипломын төсөл гүйцэтгэх төлөвлөгөө}}}\\[0.5cm]
\end{center}
%\vspace*{0.5cm}
\noindent \textbf{Дипломын төслийн сэдэв:}\\
\textbf{Монгол}: '' \ttitle '' \\
\textbf{Англи}: '' \ttitleng ''\\

\noindent \textbf{Төслийн зорилго}: Сүлжээний орчин дахь кибер аюулгүй байдлын зөрчилд хариу үзүүлэх арга, хэрэгслүүдийг судлан туршиж, ажлын зааварчилгаа боловсруулах.\\
 
\noindent \textbf{Гүйцэтгэх оюутны овог нэр}:\makebox[4cm]{ } \shortname /\studentcode/ \\
\textbf{Холбоо барих утас}: \makebox[7cm]{ }\phonenum \\
\noindent
	\begin{tabularx}{1\textwidth}{| >{\hsize=0.2\hsize}X
		| >{\hsize=2.8\hsize}Z
		| >{\hsize=0.5\hsize}Z
		| >{\hsize=0.5\hsize}Z |}
	\hline
	\multirow{2}{*}{№} &\multirow{2}{*}{Ажлын бүлэг, хэсгийн нэр} & \multirow{2}{4em}{\centering эзлэх хувь} & \multirow{2}{5em}{\centering дуусах хугацаа} \\ 
	& & & \\ \hline
	\multicolumn{4}{|l|}{\parbox[l]{13cm}{Бүлэг №1. Сүлжээний орчин дахь кибер аюулгүй байдлын онолын хэсэг}} \\  \hline
	\multirow{3}{*}{} &\multirow{3}{*}{} & \multirow{3}{*}{} & \multirow{3}{*}{} \\
	1 & \parbox[l]{9cm}{
		1.1 Кибер аюулгүй байдлын тодорхойлолт\\
		1.2 Сүлжээний кибер зөрчлийн тухай\\
		1.3 Сүлжээний кибер зөрчлийн хор уршиг\\
		1.4 Сүлжээний кибер зөрчилд үзүүлэх хариу үзүүлэх зааварчилгааны ач холбогдол\\
		1.5 Зөрчилд хариу үзүүлэх арга хэмжээ түүнтэй холбоотой ойлголтууд\\
		1.6 Бүлгийн дүгнэлт} & 20\% & III.28 \\  
	& & & \\ \hline
	\multicolumn{4}{|l|}{\parbox[l]{13cm}{Бүлэг №2. Сүлжээний орчин дахь зөрчилд хариу үзүүлэх зааварчилгаа боловсруулах арга зүйг судлах}} \\ \hline
	\multirow{3}{*}{} &\multirow{3}{*}{} & \multirow{3}{*}{} & \multirow{3}{*}{} \\
	2 & \parbox[l]{9cm}{
		2.1 Сүлжээний түгээмэл халдлагуудыг судлах \\
		2.2 Сүлжээний зөрчилд хариу үзүүлэх зааварчилгаа боловсруулах арга зүйг судлах\\
		2.3 Бүлгийн дүгнэлт} & 40\% & IV.21 \\  & & & \\ \hline
	\multicolumn{4}{|l|}{\parbox[l]{13cm}{Бүлэг №3. Сүлжээний орчин дахь кибер халдлагад хариу үзүүлэх ажлын зааварчилгаа боловсруулах нь}} \\ \hline
	\multirow{2}{*}{} &\multirow{2}{*}{} & \multirow{2}{*}{} & \multirow{2}{*}{} \\
	3 & \parbox[l]{9cm}{
		3.1 Сүлжээний зөрчилд зориулсан зааварчилгаа боловсруулах\\
		3.2 Сүлжээний зөрчлийг тодорхой кейс дээр турших\\
		3.3 Бүлгийн дүгнэлт} & 40\% & V.18 \\ & & & \\ \hline
	\multicolumn{4}{|l|}{Бүлэг №4. Ерөнхий дүгнэлт} \\  \hline
%	\multirow{2}{*}{} &\multirow{2}{*}{} & \multirow{2}{*}{} & \multirow{2}{*}{} \\
%4 & \parbox[l]{9cm}{
%	4.1 Нэмэлт бүлгүүдийг бичих\\
%	4.2 Хураангуй бичих \\
%	4.3 Дүгнэлт бичих \\
%	4.4 Дипломын бичвэрийг бүрэн боловсруулж дуусгах} & 20\% & V.25 \\  \hline
\end{tabularx}

\vspace{0.5cm}
Төлөвлөгөөг боловсруулсан оюутан: \makebox[3cm]{\dotfill} /\shortname/

%end{center}
\end{titlepage}
%---------------------ТӨЛӨВЛӨГӨӨНИЙ ХУУДСЫН ТӨГСГӨЛ ---------------------------------------------------

%---------------------ҮЗЛЭГИЙН ХУУДСЫН ЭХЛЭЛ ---------------------------------------------------
\newpage
\begin{titlepage}
	
	\small	
	\begin{center}
		\textbf{{\large ТӨГСӨЛТИЙН АЖЛЫН ҮЗЛЭГИЙН ХУУДАС}}\\
	\end{center}
	\vspace*{0.5cm}
	\noindent Оюутны код: \studentcode \\
	Оюутны нэр: \shortname \\
	Сэдвийн монгол нэр: '' \ttitle '' \\
	Сэдвийн англи нэр: '' \ttitleng ''\\
	Удирдагч багш: \supname\\
	Зөвлөгч багш: \advicenameA, \advicenameB \\
	
	\noindent	\begin{tcolorbox}[tab2,tabularx={ >{\hsize=0.2\hsize}Z| 		
			>{\hsize=0.8\hsize}Z |
			>{\hsize=1.0\hsize}Z|
			>{\hsize=0.9\hsize}Z|
			>{\hsize=2.1\hsize}Z
		},boxrule=0.9pt]
		№ & Үзлэгийн гүйцэтгэл & Гүйцэтгэлийн 30\% -с багагүй байна. & Огноо & Удирдагч \supname \hspace{0.1cm} багшийн гарын үсэг \\ \hline
		\multirow{3}{*}{1} & \multirow{3}{*}{Үзлэг-1} &    & \multirow{3}{*}{III/01-III/06} &  \\
		& & & & \\
		& & & & 
	\end{tcolorbox}
	Багшийн товч зөвлөгөө, тайлбар:
	\begin{center}
		\dotfill \\ [0.1cm]
		\dotfill \\ [0.1cm]
		\dotfill \\ [0.1cm]
		\dotfill \\ [0.1cm]
		\dotfill \\ [0.1cm]
		\dotfill \\ [0.1cm]
		\dotfill \\ [0.1cm]	
		\vspace{0.2cm}
		Үзлэг-1 хийсэн багш:\makebox[3cm]{\dotfill} /\supname/
	\end{center}
	\vspace{1cm}
	\noindent	\begin{tcolorbox}[tab2,tabularx={ >{\hsize=0.2\hsize}Z| 		
			>{\hsize=0.8\hsize}Z |
			>{\hsize=0.9\hsize}Z|
			>{\hsize=1.2\hsize}Z|
			>{\hsize=0.9\hsize}Z|
			>{\hsize=2.0\hsize}Z
		},boxrule=0.9pt]
		№ & Үзлэгийн гүйцэтгэл &Авсан оноо (10 оноо) &Гүйцэтгэлийн 50\% -с багагүй байна. & Огноо & \advicenameA \hspace{0.1cm} багшийн гарын үсэг \\ \hline
		\multirow{3}{*}{1} & \multirow{3}{*}{Үзлэг-2} &  &  & \multirow{3}{*}{IV/15-IV/19} &  \\
		& & & & & \\
		& & & & &
	\end{tcolorbox}
	Багшийн товч зөвлөгөө, тайлбар:
	\begin{center}
		\dotfill \\ [0.1cm]
		\dotfill \\ [0.1cm]
		\dotfill \\ [0.1cm]
		\dotfill \\ [0.1cm]
		\dotfill \\ [0.1cm]
		\dotfill \\ [0.1cm]
		\dotfill \\ [0.1cm]	
		\vspace{0.2cm}
		Үзлэг-2 хийсэн багш:\makebox[3cm]{\dotfill} /\advicenameA/
	\end{center}
\end{titlepage}
\newpage
\begin{titlepage}
\small	
	\begin{center}
	\textbf{{\large ТӨГСӨЛТИЙН АЖЛЫН ҮЗЛЭГИЙН ХУУДАС}}\\
\end{center}
\vspace*{0.5cm}
\noindent Оюутны код: \studentcode \\
Оюутны нэр: \shortname \\
Сэдвийн монгол нэр: '' \ttitle '' \\
Сэдвийн англи нэр: '' \ttitleng ''\\
Удирдагч багш: \supname\\
Зөвлөгч багш: \advicenameA, \advicenameB \\
\noindent	\begin{tcolorbox}[tab2,tabularx={ >{\hsize=0.2\hsize}Z| 		
		>{\hsize=0.8\hsize}Z |
		>{\hsize=0.9\hsize}Z|
		>{\hsize=1.2\hsize}Z|
		>{\hsize=0.9\hsize}Z|
		>{\hsize=2.0\hsize}Z
	},boxrule=0.9pt]
	№ & Үзлэгийн гүйцэтгэл &Авсан оноо (10 оноо) &Гүйцэтгэлийн 70\% -с багагүй байна. & Огноо & \advicenameB \hspace{0.1cm} багшийн гарын үсэг \\ \hline
	\multirow{3}{*}{1} & \multirow{3}{*}{Үзлэг-3} &  &  & \multirow{3}{*}{VI/29-V/03} &  \\
	& & & & & \\
	& & & & &
\end{tcolorbox}
Багшийн товч зөвлөгөө, тайлбар:
\begin{center}
	\dotfill \\ [0.1cm]
	\dotfill \\ [0.1cm]
	\dotfill \\ [0.1cm]
	\dotfill \\ [0.1cm]
	\dotfill \\ [0.1cm]
	\dotfill \\ [0.1cm]
	\dotfill \\ [0.1cm]	
	\vspace{0.2cm}
	Үзлэг-3 хийсэн багш:\makebox[3cm]{\dotfill} /\advicenameB/
\end{center}		
\vspace{1cm}
\noindent	\begin{tcolorbox}[tab2,tabularx={ >{\hsize=0.2\hsize}Z| 		
		>{\hsize=0.8\hsize}Z |
		>{\hsize=1.0\hsize}Z|
		>{\hsize=0.9\hsize}Z|
		>{\hsize=2.1\hsize}Z
	},boxrule=0.9pt]
	№ & Үзлэгийн гүйцэтгэл & Гүйцэтгэлийн 90\% -с багагүй байна. & Огноо & Удирдагч \supname \hspace{0.1cm} багшийн гарын үсэг \\ \hline
	\multirow{3}{*}{1} & \multirow{3}{*}{Үзлэг-4} &    & \multirow{3}{*}{V/13-V/17} &  \\
	& & & & \\
	& & & & 
\end{tcolorbox}
\vspace{0.5cm}
\noindent	\begin{tcolorbox}[tab2,tabularx={ >{\hsize=0.2\hsize}Z| 		
		>{\hsize=1.6\hsize}Z |
		>{\hsize=0.8\hsize}Z|
		>{\hsize=1.4\hsize}Z
	},boxrule=0.9pt]
	№ & Удирдагч \supname \hspace{0.1cm} багшийн үнэлгээ (30 оноо) & Огноо & Удирдагч багшийн гарын үсэг \\ \hline
	\multirow{3}{*}{1} &    & \multirow{3}{*}{V/17} &  \\
	& & & \\
	& & & 
\end{tcolorbox}
\begin{center}
\vspace{0.5cm}
	Удирдагч  багш:\makebox[3cm]{\dotfill} /\supname/ \\[0.5 cm]
	\textit{\footnotesize Жич: Удирдагч багш өөрийн үнэлгээгээ 30 хүртэл оноогоор өгөх ба үнэлгээ тавьсан хуудсыг оюутанд буцааж өгөлгүй төгсөлтийн нарийн бичгийн даргад хураалгана уу.}
\end{center}
\end{titlepage}

%---------------------ҮЗЛЭГИЙН ХУУДСЫН ТӨГСГӨЛ ---------------------------------------------------



%---------------------ГҮЙЦЭТГЭЛИЙН ХУУДСЫН ЭХЛЭЛ ---------------------------------------------------
\newpage

\begin{titlepage}

\begin{center}

\vspace*{2cm}
\textbf{{\large ТӨГСӨЛТИЙН АЖЛЫН ЯВЦ}}\\[0.5cm]

\begin{tabularx}{1\textwidth}{| >{\hsize=0.1\hsize}Z
		| >{\hsize=2.5\hsize}X
		| >{\hsize=0.6\hsize}Z
		| >{\hsize=0.8\hsize}Z |}
	\hline
	\multirow{2}{*}{№} & \multirow{2}{*}{Хийж гүйцэтгэсэн ажил} & Биелсэн & Удирдагчийн \\
	  & & хугацаа & гарын үсэг \\ \hline
	1 & {Бүлэг №1. Сүлжээний орчин дахь кибер аюулгүй байдлын онолын хэсэг} & 2022-3-28 &  \\ \hline
	2 & {Бүлэг №2. Сүлжээний орчин дахь зөрчилд хариу үзүүлэх зааварчилгаа боловсруулах арга зүйг судлах} & 2022-4-21 &  \\ \hline
	3 & {Бүлэг №3. Сүлжээний орчин дахь кибер халдлагад хариу үзүүлэх ажлын зааварчилгаа боловсруулах нь} & 2022-5-18 &  \\ \hline
	4 & {Бүлэг №4. Ерөнхий дүгнэлт}    & 2022-5-25 &  \\ \hline
\end{tabularx}

\vspace{1cm}
Ажлын товч дүгнэлт \\[0.5cm]

\dotfill \\ [0.2cm] 
\dotfill \\ [0.2cm]
\dotfill \\ [0.2cm]
\dotfill \\ [0.2cm]
\dotfill \\ [0.2cm]
\dotfill \\ [0.2cm]
\dotfill \\ [0.5cm]

Удирдагч: \makebox[3cm]{\dotfill} /\supname/ \\

\vspace{2cm}
ЗӨВШӨӨРӨЛ \\[0.5cm]
Оюутан \shortname --н бичсэн төгсөлтийн ажлыг УШК-д хамгаалуулахаар тодорхойлов.\\[0.5cm]
Салбарын эрхлэгч: \makebox[3cm]{\dotfill} /\chairname/
\end{center}

\end{titlepage}
%---------------------ГҮЙЦЭТГЭЛИЙН ХУУДСЫН ТӨГСГӨЛ ---------------------------------------------------

%---------------------ШҮҮМЖИЙН ХУУДСЫН ЭХЛЭЛ ---------------------------------------------------
\newpage

\begin{titlepage}
\begin{center}

{\scshape\Large \univname\par} % Их сургуулийн нэр
{\scshape\large \facname\par}\vspace{1cm} % Их сургуулийн нэр

\textbf{{\Large ШҮҮМЖИЙН ХУУДАС}}\\[1cm]

\end{center}

\normalsize

\deptname --н салбарын төгсөх курсийн оюутан \shortname -н ''\ttitle'' сэдэвт төгсөлтийн ажлын шүүмж.

\begin{enumerate}
\item Төслөөр дэвшүүлсэн асуудал, үүнтэй холбоотой онолын материал уншиж судалсан байдал. Энэ талаар хүмүүсийн хийсэн судалгаа, түүний үр дүнг уншиж тусгасан эсэх.
\begin{center}
\dotfill \\[0.1cm]
\dotfill \\[0.1cm]
\dotfill \\[0.1cm]
\dotfill \\[0.1cm]
\dotfill \\[0.1cm]
\dotfill \\[0.1cm]
\dotfill \\[0.4cm]
\end{center}
\item Төслийн ерөнхий агуулга. Шийдсэн зүйлүүд, хүрсэн үр дүн. Өөрийн санааг гарган, харьцуулалт хийн, дүгнэж байгаа чадвар.
\begin{center}
\dotfill \\[0.1cm]
\dotfill \\[0.1cm]
\dotfill \\[0.1cm]
\dotfill \\[0.1cm]
\dotfill \\[0.1cm]
\dotfill \\[0.1cm]
\dotfill \\[0.4cm]
\end{center}
\item Эмх цэгцтэй, стандарт хангасан өөрөөр хэлбэл диплом бичих шаардлагуудыг биелүүлсэн эсэх. Төсөлд анзаарагдсан алдаанууд, зөв бичгийн болон өгүүлбэр зүйн гэх мэт /Хуудас дугаарлагдаагүй, зураг хүснэгтийн дугаар болон тайлбар байхгүй, шрифт хольсон, хувилсан зүйл ихээр оруулсан/.
\begin{center}
\dotfill \\[0.1cm]
\dotfill \\[0.1cm]
\dotfill \\[0.1cm]
\dotfill \\[0.1cm]
\dotfill \\[0.1cm]
\end{center}
\end{enumerate}
\end{titlepage}

\newpage

\begin{titlepage}
\begin{enumerate}
\item[4.] Төслөөр орхигдуулсан болон дутуу болсон зүйлүүд. Цаашид анхаарах хэрэгтэй зүйлүүд.
\begin{center}
\dotfill \\[0.1cm]
\dotfill \\[0.1cm]
\dotfill \\[0.1cm]
\dotfill \\[0.1cm]
\dotfill \\[0.1cm]
\dotfill \\[0.1cm]
\dotfill \\[0.4cm]
\end{center}
\item [5.] Төслийн талаар онцолж тэмдэглэх зүйлүүд.
\begin{center}
\dotfill \\[0.1cm]
\dotfill \\[0.1cm]
\dotfill \\[0.1cm]
\dotfill \\[0.1cm]
\dotfill \\[0.1cm]
\dotfill \\[0.1cm]
\dotfill \\[0.4cm]
\end{center}
\item [6.] Ерөнхий оноо. (30 оноо)
\begin{center}
\dotfill \\[1cm]
\end{center}
\end{enumerate}
Шүүмж бичсэн: \makebox[3cm]{\dotfill} /\readname/ \\[0.5cm]
Ажлын газар: \dotfill \\[0.5cm]
Хаяг (Утас) \makebox[5cm]{\dotfill}
\end{titlepage}

%---------------------ШҮҮМЖИЙН ХУУДСЫН ТӨГСГӨЛ --------------------------------------------------- % Төлөвлөгөө, гүйцэтгэл, шүүмжийн хуудас
\include{FrontBackMatter/Declaration} % Мэдэгдлийн хуудас
%-------------------------------------------------------------------------------
%	INTRODUCTION PAGE
%-------------------------------------------------------------------------------
\section*{Зорилго}
Прокси дахин шифрлэлэх схем судалж, аюулгүй файл хуваалцах систем туршилтын загвар хөгжүүлэх.

\section*{Зорилт}
Дээрх зорилгыг хэрэгжүүлэхийн тулд дараах зорилтуудыг дэвшүүлж байна.
\begin{itemize}
    \item Шифрлэлтийн схемүүдийг судлах
    \item Файл шифрлэх аргуудыг судлах
    \item Хөгжүүлэхэд шаардлагатай хэрэглэгдэхүүнийг судлах
    \item Аюулгүй файл хуваалцах үйлчилгээний загвар гарах
    \item Хөгжүүлсэн системийг туршиж, ажиллуулах
\end{itemize}

 % Удиртгал хэсэг

%-------------------------------------------------------------------------------
%	ABSTRACT PAGE
%-------------------------------------------------------------------------------

\addchaptertocentry{Хураангуй} % Хураангуйг гарчигт нэмэх

\begin{center}
    {\scshape\Large \univname\par} % Их сургуулийн нэр
    {\scshape\large \facname\par}\vspace{0.5cm} % Их сургуулийн нэр
    {\huge\textbf{{Хураангуй}} \par}
    \bigskip
    {\Large{\ttitle} \par} % Тезисийн нэр
    \bigskip

    {\normalsize \shortname \par} % Зохиогчийн нэр
    \addressname
\end{center}

\textit{\textbf{Түлхүүр үгс: \keywordnames}}
\bigskip

Прокси дахин шифрлэх схем нь өгөгдөл эзэмшигч өөрийн нийтийн түлхүүрээр шифрлэгдсэн итгэмжлэгдсэн хүнд шифрийг тайлахгүйгээр гурав дахь ч хагас итгэмжлэгдсэн тал шифрийг дахин шифрлэж итгэмжлэгдсэн хүн тайлах боломжийг олдог.

Прокси дахин шифрлэх схемийг ашиглан файл хуваалцах туршилтын систем хөгжүүлэлтийг хийж гүйцэтгэв. % Ажлын хураангуй
%-------------------------------------------------------------------------------
%	ACKNOWLEDGEMENTS
%-------------------------------------------------------------------------------

\begin{acknowledgements}
\addchaptertocentry{\acknowledgementname}

Энэхүү дипломын ажлыг бичихэд туслалцаа үзүүлсэн удирдагч багш \\
Н.Чулуунбаатар болон ШУТИС-ийн Мэдээлэл холбоо технологийн сургуулийн Электроникийн салбарын багш нарт талархсанаа илэрхийлье. 


\end{acknowledgements}

 % Талархлын хуудас
%-------------------------------------------------------------------------------
%	ABBREVIATIONS
%-------------------------------------------------------------------------------

\begin{abbreviations}{ll} % Товчлолын жагсаалт оруулах (хоёр багатай хүснэгт)
\addchaptertocentry{\abbrevname}

\textbf{CPU} & \textbf{C}entral \textbf{P}processing \textbf{U}nit\\
\textbf{UML} & \textbf{U}nified \textbf{M}odelling \textbf{L}anguage\\
\textbf{GPU} & \textbf{G}raphic \textbf{P}rocessing \textbf{U}nit\\
\textbf{ННТ} & \textbf{Н}исэгчгүй \textbf{Н}исэх \textbf{Т}өхөөрөмж\\
\textbf{ЦДҮС} & \textbf{Ц}ахилгаан \textbf{Д}амжуулах \textbf{Ү}ндэсний \textbf{С}үлжээ\\
\textbf{ЦДАШ} & \textbf{Ц}ахилгаан \textbf{Д}амжуулах \textbf{А}гаарын \textbf{Ш}угам\\
\textbf{NLP} & \textbf{N}atural \textbf{L}anguage \textbf{P}rocessing\\
\textbf{CNN} & \textbf{C}onvolutional \textbf{N}eural \textbf{N}etworks\\
\textbf{ReLU} & \textbf{R}ectified \textbf{L}inear \textbf{U}nit\\


\end{abbreviations}

 % Товчилсон үгс

%-------------------------------------------------------------------------------
%	LIST OF CONTENTS/FIGURES/TABLES PAGES
%-------------------------------------------------------------------------------

\tableofcontents % Гарчиг хэвлэх
\listoffigures % Зургийн жагсаалт хэвлэх
\listoftables % Хүснэгтийн жагсаалт хэвлэх

%-------------------------------------------------------------------------------
%	THESIS CONTENT - CHAPTERS
%-------------------------------------------------------------------------------

\mainmatter % Хуудасны тоон (1,2,3...) дугаарлалт эхлэнэ

\pagestyle{thesis} % Хуудасны тогойг "thesis" загвар руу буцаах
\myformat % Бүлгийн нэрийг тусгай хуудсанд хэвлэх

% Тезисийн бүлгүүдийг Chapters хавтаснаас бие даасан файл байдлаар оруулах
% \pagecolor{Beige}
% Бүлэг 1

\chapter{Өгөгдөл хуваалцах үйлчилгээний тухай} % Бүлгийн нэр
\label{Chapter1} % Энэ бүлэг рүү ишлэл хийх бол \ref{Chapter1} командыг ашигла 
\pagecolor{white}
%-------------------------------------------------------------------------------

% Агуулгад ашигласан хэвшүүлэлтийн зарим командын тодорхойлолт
\newcommand{\keyword}[1]{\textbf{#1}}
\newcommand{\tabhead}[1]{\textbf{#1}}
\newcommand{\code}[1]{\texttt{#1}}
\newcommand{\file}[1]{\texttt{\bfseries#1}}
\newcommand{\option}[1]{\texttt{\itshape#1}}

%-------------------------------------------------------------------------------
%	SECTION 1
%-------------------------------------------------------------------------------

\section{Өгөгдөл хуваалцах үйлчилгээ}
Орчин үеийн мэдээллийн технологийн эрин үед байгууллагуудад мэдээлэл солилцох олон шалтгаан, хэрэгцээ байдаг - ажилчдад алсаас ажиллах боломжийг олгох, үйл ажиллагааны үр ашгийг дээшлүүлэх, эсвэл гуравдагч талын үйлдвэрлэгчидтэй хамтран ажиллах зэрэг олон шалтгаан бий.

\textbf{Өгөгдөл хуваалцах ямар технологиуд}\\
Өгөгдөл хуваалцах олон технологи байдаг. Зарим технологиудаас дурдвал.

\begin{itemize}
    \item \textbf{Өгөгдлийн агуулах (Data warehousing)} нь нэг буюу хэд хэдэн ялгаатай эх сурвалжийг нэгтэгсэн төвлөрсөн агуулах юм. Архитектур нь шатлалаас бүрддэг. Дээд давхарга нь тайлагнах, дүн шинжилгээ хийх, үр дүнг харуулдаг front-end клиент юм. Дунд шат нь өгөгдөлд хандах, дүн шинжилгээ хийхэд ашигладаг аналитик механизмаас бүрдэнэ. Доод шат нь өгөгдлийг ачаалах, хадгалах өгөгдлийн сангийн сервер юм. Дээд болон дунд түвшний програмууд нь доод давхаргад хадгалагдсан нийтлэг өгөгдлийн багцыг хуваалцах боломжтой.
    
    \item \textbf{Хэрэглээний программчлалын интерфэйс (API)} нь програм хангамжийн хоёр бүрэлдэхүүн хэсэг нь тодорхой протоколуудыг ашиглан хоорондоо харилцах боломжийг олгодог механизм юм. Интерфэйсийг  хоёр програмын хоорондох үйлчилгээний тохиролцоо гэж үзэж болно. Энэхүү тохиролцоо нь хэрхэн харилцах хүсэлт болон хариултыг тодорхойлдог. Хандалтыг нарийн тодорхойлж болдог ба хэрэглэгчид яг ямар өгөгдөл хүсч болохыг зааж өгдөг.
    
    \item \textbf{Холбооны сургалт (Federated learning)} нь тархсан өгөгдлийг багц дээр хиймэл оюун ухааныг сургах боломжийг олгодог. Бүх өгөгдлийг нэг дор цуглуулж нэгтэхийн оронд тус тусдаа төхөөрмж дээр хадгалж зөвхөн загварийн шинэчлэлтүүдийг төв сервер рүү илгээдэг.

    \item \textbf{Блокчейн технологи} нь сүлжээн дотор ил тод мэдээлэл солилцох боломжийг олгодог өгөгдлийн сангийн дэвшилтэт механизм юм. Өгөгдлийг гинжин хэлхээнд холбосон блокуудад хадгалдаг. Сүлжээнээс зөвшилцөлгүйгээр гинжийг устгах эсвэл өөрчлөх боломжгүй.

    \item \textbf{Өгөгдөл солилцох платформууд}
    
    Нээлттэй өгөгдлийн платформууд нь өөр өөр өгөгдлийн багцийг нийтийн хэрэгцээнд ашиглах боложийг ологдог. Ихэвчлэн өгөгдлийн менежмент, өгөгдлийн аюулгүй байдал, өгөгдөл нэгтгэх, өгөгдөл хуваалцах, хамтран ажиллах зэрэг олон төрлийн функцуудыг санал болгодог.
\end{itemize}

%-------------------------------------------------------------------------------
%	SECTION 2
%-------------------------------------------------------------------------------

\section{Өгөгдлийн аюулгүй байдал}

Өгөгдлийн аюулгүй байдал гэдэг нь дижитал мэдээллийг зөвшөөрөлгүй хандах, өөрчлөх, хулгайлахаас хамгаалах үйл ажиллагаа юм. Физик төхөөрөмжийн хамгаалалтаас эхлээд хандалтын удирдлаг, програм хангамжийн логик аюулгүй байдал мэдээллийн аюулгүй байдалын бүх талыг хамарсан ойлголт юм.

\textbf{Ягаад өгөгдлийн аюулгүй байдал чухал вэ?}

Таний нууц эмзэг мэдээлэл санхүүгийн чадамж бичиг баримт зэргийг буруу зорилгоор ашиглах аюултай.

Байгуулгын хувьд хэрэглэгчдийнхээ мэдээлэл өгөгдлийг алдаж буруу гарт орохоос сэргийлэж хуулийн дагуу хамгаалах ёстой. Мөн тухайн байгуулга нь хакдуулах мэдээлэлээ алдах нь нэр хүнд нь халтай ба хэрэглэгчдийн итгэлийг алдах аюултай.

\textbf{Өгөгдлийн аюулгүй байдлын төрлүүд}
\begin{itemize}
    \item \textbf{Шифрэлэлт} нь түлхүүр нууц үггүйгээр өгөгдлийг унших боломжгүй бологдог ба криптографын алгоритмуудыг ашиглан энгийн текстийг шифрлэх үйл явц юм. Энэ нь халдагчид өгөгдөлд нэвтэрсэн байсан ч зохих итгэмжлэлгүйгээр үүнийг уншиж чадахгүй гэдгийг баталгаажуулахад тусалдаг.
    \item \textbf{Хандалтын удирдлага} нь нууц өгөгдөлд хэн хандах эрхтэй болохыг тэдний үүрэг, зөвшөөрлийн түвшинд үндэслэн хязгаарладаг. Үүнд нууц үг, биометрийн баталгаажуулалт, хамгаалалтын токен зэрэг арга хэмжээ багтана.
    \item \textbf{Нөөцлөх, сэргээх} үйл явц нь аюулгүй байдлын зөрчил эсвэл өгөгдөл алдагдсан тохиолдолд сэргээх боломжтой байхын тулд мэдээллийн хуулбарыг үүсгэх, хадгалах явдал юм.
    \item \textbf{Физик аюулгүй байдал} нь өгөгдөл хадгалах төхөөрөмж болон физик хандалтыг хамгаалахын тулд түгжээтэй хаалга, хамгаалалтын камер зэрэг физик хамгаалалтын арга хэмжээг ашигладаг.
    \item \textbf{Өгөгдөл устгах} Өгөгдлийг устгах нь хамгийн аюулгүй хэдий дахин ашиглах боломжгүй. Ихэвчлэн дахин ашиглахгүй өгөгдлийн дарж бичих зэргээр устгадаг.
    \item \textbf{Өгөгдлийн далдлах} нь нууц мэдээллийг анхны өгөгдлийн бүтцийг хадгалан зөвшөөрөлгүй хэрэглэгчдэд ашиглах боломжгүй болгож буй хуурамч мэдээллээр солих явдал юм.
\end{itemize}

\textbf{Аюулгүй өгөгдөл хуваалцах}


%-------------------------------------------------------------------------------
%	SECTION 3
%-------------------------------------------------------------------------------

\section{Шифрлэх схемүүд}
Өгөгдлийг хэрхэн найдвартай нууцлаж хамгаалах нь чухал болсон. Зөвхөн шифрлэхээс гадна үүнийг схемчилж илүү хурдан өөр өөрсдийн давуу талтай схемүүдмйг хөгжүүлж гаргаж ирсэн.

\textbf{Танилтад суурилсан шифрлэлт (IBE)} 

Нийтийн түлхүүрийн оронд өөрийн хувийн мэдээллийг ашиглан өгөгдлийг шифрлэх, тайлах боломжийг олгодог нийтийн түлхүүрийн шифрлэлтийн нэг төрөл юм. IBE-ийг хэрэглэгчдийг таних тэмдэгээр нь мэддэг тохиолдолд аюулгүй өгөгдөл хуваалцахад ашиглаж болно.

\textbf{Шинж чанарт суурилсан шифрлэлэт (ABE)}

Энэ нь нас, албан тушаал, байгууллагын үүрэг зэрэг урьдчилан тодорхойлсон шинж чанарт үндэслэн өгөгдөлд хандах боломжийг олгодог шифрлэлтийн төрөл юм. ABE нь өгөгдөлд хандах хандалтыг нарийн хянахад ашиглагдаж болох ба зарим шинж чанарууд дээр үндэслэн хандалт олгосон хувилбаруудад ашиглаж болно.

\textbf{Гомоморф шифрлэлт (HE)}

Энэ нь шифрлэгдсэн өгөгдлийг эхлээд тайлахгүйгээр тооцоолол хийх боломжийг олгодог шифрлэлтийн төрөл юм. Тооцоолол хийх боломжийг олгохын зэрэгцээ өгөгдлийг нууцлах шаардлагатай тохиолдолд HE-г аюулгүй өгөгдөл боловсруулахад ашиглаж болно.

\textbf{Secure multiparty computation (MPC)}

Энэ талууд өөрсдийн оролтыг бие биедээ ил гаргахгүйгээр хувийн оролт дээрээ функцийг хамтран тооцоолох боломжийг олгодог криптографийн арга юм. Мэдээллийн нууцлалыг хадгалах, олон тал хамтран ажиллах шаардлагатай тохиолдолд MPC-ийг аюулгүй өгөгдөл боловсруулахад ашиглаж болно.

\textbf{Прокси дахин шифрлэлт (PRE)}

%-------------------------------------------------------------------------------
%	SECTION 4
%-------------------------------------------------------------------------------

\section{Файл шифрлэх аргууд}

\section{Шифрлэлт, түүний ач холбогдол, ангилал, хэрэглээ}
Мэдээллийн аюулгүй байдал үндсэн гурван зарчимыг тэнцвэртэй хангахыг зоридог. 
\begin{itemize}
    \item \textbf{Нууцлаг байдал (Confidentiality)}: Мэдээлэлийг нууц хэвээр нь хамгаалж үлдэх. Санаатай болон санамсаргүй мэдээллийг зөвшөөрөлгүй хуваалцах тараахаас сэргийлэх.
    \begin{itemize}
        \item Өгөгдлийн нууцлал (Data1 confidentiality)
        \item хувийн нууц (Privacy)
    \end{itemize}
    \item \textbf{Бүрэн бүтэн байдал (Integrity)}: Өгөгдөлд үнэн зөв найдвартай гадны нөлөө ороогүйг шалгах, бүрэн бүтэн хадаглах. 
    \begin{itemize}
        \item Өгөгдлийн бүрэн бүтэн байдал (Data integrity)
        \item Системийн бүрэн бүтэн байдал (System integrity) 
    \end{itemize}
    \item \textbf{Хүртээмжтэй байдал (Availability):} Тухайн системийн хэрэглэгчид хүртээмжтэй байх.
\end{itemize}

\begin{figure}[ht]
\centering
\includegraphics[scale=0.7]{Figures/cia_traid}
\caption[CIA гурвалжин]{CIA гурвалжин}
\label{fig:CIA}
\end{figure}

Криптографд шифрлэлт нь энгийн текстийг (жишээ нь, эх мессеж) шифр текст (жишээ нь, шифрлэгдсэн эсвэл кодлогдсон мессеж) болгон хувиргахад ашигладаг алгоритм юм. Шифрлэлтийн зорилго нь мессежийг түлхүүргүй хүн унших боломжгүй болгох явдал юм.

Мэдээлэл болон өгөгдлийг шифрлэлт хийснээр нууцлаг байдлыг хангах хамгийн том давуу тал юм. Бүрэн бүтэн байдал болон хүртээмжтэй байдлыг ч шифрлэлт нь хангах боломжтой.
Шифрлэлт ерөнхийд нь гурав ангилна.
\begin{itemize}
    \item \textbf{Тэгш хэмт шифрлэлт (symmetric)} нь шифрийг тайлах болон шифрлэхдээ нэг түлхүүр ашигладаг. Уламжлалт шифрлэлт гэх нь бий. Уламжлалт (компьютероос өмнөх үе) тэгш хэмтэй шифрүүд нь орлуулах эсвэл шилжүүлэх аргыг ашигладаг. Орлуулах арга нь энгийн текстийн элементүүд (тэмдэгтүүд, битүүд) шифр текстийн элементүүдэд солино оруулж тавина. Шилжүүлэх техник нь энгийн текстийн элементүүдийн байрлалыг системтэйгээр шилжүүлдэг. 
    Тэгш хэмт шифрлэлт нь хоёр төрөлтэй.
    \begin{itemize}
        \item Урсгал шифрлэлт (Stream шифрлэлт) RC4 болон ChaCha20 гэх мэт.
        \item Блок шифрлэлт (Block шифрлэлт) AES, DES, болон 3DES гэх мэт.
    \end{itemize}
    \item \textbf{Тэгш бус шифрлэлт (asymmetric)} нь нийтийн болон хувийн хоёр түлхүүртэй. Нийтийн түлхүүр нь нийтэд нээлтэй байдаг. Хувийн түлхүүрийг эмзэгшигч нь нууцалж алдахгүй байх ёстой. 
    \item \textbf{Хэш (Hash)} функц нь хувьсах урттай мессежийг тогтмол урттай хэш утга шифрлэдэг. Ихэнх хэш функц нь шахалтын алгоритм ашигладаг.
\end{itemize}

Криптограф нь дамжуулалтын явцад мессежийг хөндлөнгөөс өөрчлөлт ороогүй эсэхийг шалгаж бүрэн бүтэн байдлыг хангадаг. Хэш, мессежийг баталгаажуулалтын код (MACs), тоон гарын үсэг (Digital Signatures) ашигладаг байдаг.

\textbf{Мессежийн баталгаажуулалтын код (MACs)} нь авсан өгөгдөл нь илгээсэнтэй яг таарч (өөрчлөлт оруулах, устгах) мөн илгээгчийн баталгаажуулдаг.
Нууц түлхүүр ашигдаг. MAC нь хувьсах урттай мессежийг нууц түлхүүр болгон авч, баталгаажуулах код үүсгэдэг. MAC нь хэш функц болон тэгш хэмт блок шифрлэлтийг ашиглдаг.

\textbf{Тоон гарын үсэг (Digital Signatures)}

Ихэвчлэн шифрлэгдсэн мессэж, энгийн мессежийн хэшийг бүтээгчийн хувийн түлхүүрийн хэшийг авч харьцуулж баталгаажуулдаг.

%-------------------------------------------------------------------------------
%	SECTION 5
%-------------------------------------------------------------------------------

\section{Бүлгийн Дүгнэлт}
Энэ бүлэгт орчин үеийн шифрлэлтийн схеммүүдмйг судалж прокси дахин шифрлэлэт нь бусад схеммүүдээс ямар давуу тал сул талыг судалж харицуулсан. Системийн хөгжүүлэлт ерөнхий загварийг гаргаж юу хэрэгтэй сангуудыг ашиглан системийн хөгжүүлэлтыг хийж элсэн.
% \pagecolor{Beige}
% Бүлэг 2

\chapter{А.Эрдэнэбаатарын зөвлөмж} % Зарим нэг зөвлөмж

\label{Chapter2} % Энэ бүлэг рүү ишлэл хийх бол \ref{Chapter2} командыг ашигла 
\pagecolor{white}
%-------------------------------------------------------------------------------
%	SECTION 1
%-------------------------------------------------------------------------------

% \pagecolor{Beige}
% Бүлэг 1

\chapter{Прокси дахин шифрлэлтэд суурилсан файл хуваалцах систем хөгжүүлэх}

\label{Chapter3} % Энэ бүлэг рүү ишлэл хийх бол \ref{Chapter1} командыг ашигла 
\pagecolor{white}

%-------------------------------------------------------------------------------
%	SECTION 1
%-------------------------------------------------------------------------------
\section{Системийн шаардлага}
Систем нь түлхүүр үүсгэх файл хадгалах сервер, өгөгдөл эзэмшигч, өгөгдөл хэрэглэгч гэсэн үндсэн гурван хэсгээс тогтоно. Өгөгдөл эзэмшигч бүртгэл үүсгэж өөрийн нийтийн түлхүүр болон хувийн түлхүүрийг үүсгэж авна. Файл шифрлэх болон тайлах үйлдэлийг өөрийн төхөөрөмж дээр үйлдэнэ.

\textbf{Системийн оролцогч}
\begin{itemize}
    \item Хэрлэгч
\end{itemize}

\textbf{Системийн тоглогч}
\begin{itemize}
    \item Өгөгдөл эзэмшигч
    \item Өгөгдөл хэрэглэгч
\end{itemize}

\subsection*{Функцийн шаардлага}
Өгөгдөл эзэмшигчийн функционал шаардлага:
\begin{itemize}
    \item Системд өөрийн бүртгэлийг үүсгэх
    \item Файл оруулах, шифрлэх
    \item Файлыг тайлах хэрлэгч сонгох
    \item Шифрлсэн файлыг хуваалцсан хэрлэгчдийн жагсаалт
\end{itemize}
Өгөгдөл хэрэглэгчийн функционал шаардлага:
\begin{itemize}
    \item Системд өөрийн бүртгэлийг үүсгэх
    \item Өөрт хуваацлсан файлын жагсаалт
    \item Файлыг татаж авах
    \item Шифрлсэн файлыг тайлах
\end{itemize}

\subsection*{Функцийн бус шаардлага}
\begin{enumerate}
    \item Файлыг шифрлэх, шифрийг тайлах хурдан гүйцэтгэдэг байх
    \item Хэрэглэгчийн интерфэйс ойлгомжтой энгийн байх.
    \item Прокси серверт файлыг шифрлсэн байдлаар хадгалах, хуваалцах
    \item Өөрийн бүртгэлийг ашиглаж нэвтрэх
    \item Хувийн түлхүүрийг хэрлэгчийн төхөөрөмж дээр авч явах
\end{enumerate}

\subsection*{Юзкейс диаграмм}
Системийн хэрлэгчид ямар үйлдлүүдийг системд хийж болохыг харуулсан хэрэглээний диаграммыг харууллаа.
\begin{figure}[ht]
    \centering
    \includegraphics[scale=0.6]{Figures/usecase.drawio.png}
    \caption[Usecase diagram]{Юзкейс диаграмм}
    \label{fig:usecase}
\end{figure}
\subsection*{Юзкейс тодорхойлолт}

\begin{table}
    \caption{Бүртгүүлэх юзкейсийн тодорхойлолт}
    \label{tab:treatments}
    \footnotesize
    \centering
    \begin{tabularx}{\textwidth}{|>{\hsize=0.3\hsize}X|>{\hsize=0.7\hsize}X|}
        \hline
        \multicolumn{2}{|c|}{Бүртгүүлэх} \\
        \hline
        ID & 1 \\
        \hline
        Тодорхойлолт & \\
        \hline
        Өмнөх нөхцөл & \\
        \hline
        Үндсэн урсгал & \\
        \hline
        Дараах нөхцөл & \\
        \hline
      \end{tabularx}
\end{table}

\begin{table}
    \caption{Нэвтрэх юзкейсийн тодорхойлолт}
    \label{tab:treatments}
    \footnotesize
    \centering
    \begin{tabularx}{\textwidth}{|>{\hsize=0.3\hsize}X|>{\hsize=0.7\hsize}X|}
        \hline
        \multicolumn{2}{|c|}{Нэвтрэх} \\
        \hline
        ID & 2 \\
        \hline
        Тодорхойлолт & \\
        \hline
        Өмнөх нөхцөл & \\
        \hline
        Үндсэн урсгал & \\
        \hline
        Дараах нөхцөл & \\
        \hline
      \end{tabularx}
\end{table}

\begin{table}
    \caption{Файл шифрлэх юзкейсийн тодорхойлолт}
    \label{tab:treatments}
    \footnotesize
    \centering
    \begin{tabularx}{\textwidth}{|>{\hsize=0.3\hsize}X|>{\hsize=0.7\hsize}X|}
        \hline
        \multicolumn{2}{|c|}{Файл шифрлэх} \\
        \hline
        ID & 3 \\
        \hline
        Тодорхойлолт & \\
        \hline
        Өмнөх нөхцөл & \\
        \hline
        Үндсэн урсгал & \\
        \hline
        Дараах нөхцөл & \\
        \hline
      \end{tabularx}
\end{table}

\begin{table}
    \caption{Шифрлсэн файл серверт байршуулах юзкейсийн тодорхойлолт}
    \label{tab:treatments}
    \footnotesize
    \centering
    \begin{tabularx}{\textwidth}{|>{\hsize=0.3\hsize}X|>{\hsize=0.7\hsize}X|}
        \hline
        \multicolumn{2}{|c|}{Шифрлсэн файл серверт байршуулах} \\
        \hline
        ID & 4 \\
        \hline
        Тодорхойлолт & \\
        \hline
        Өмнөх нөхцөл & \\
        \hline
        Үндсэн урсгал & \\
        \hline
        Дараах нөхцөл & \\
        \hline
      \end{tabularx}
\end{table}

\begin{table}
    \caption{Файл хуваалцах тодорхойлолт}
    \label{tab:treatments}
    \footnotesize
    \centering
    \begin{tabularx}{\textwidth}{|>{\hsize=0.3\hsize}X|>{\hsize=0.7\hsize}X|}
        \hline
        \multicolumn{2}{|c|}{Файл хуваалцах} \\
        \hline
        ID & 5 \\
        \hline
        Тодорхойлолт & \\
        \hline
        Өмнөх нөхцөл & \\
        \hline
        Үндсэн урсгал & \\
        \hline
        Дараах нөхцөл & \\
        \hline
      \end{tabularx}
\end{table}

\begin{table}
    \caption{Файл татах тодорхойлолт}
    \label{tab:treatments}
    \footnotesize
    \centering
    \begin{tabularx}{\textwidth}{|>{\hsize=0.3\hsize}X|>{\hsize=0.7\hsize}X|}
        \hline
        \multicolumn{2}{|c|}{Файл татах} \\
        \hline
        ID & 6 \\
        \hline
        Тодорхойлолт & \\
        \hline
        Өмнөх нөхцөл & \\
        \hline
        Үндсэн урсгал & \\
        \hline
        Дараах нөхцөл & \\
        \hline
      \end{tabularx}
\end{table}

\begin{table}
    \caption{Шифрлсэн файлын тайлах юзкейсийн тодорхойлолт}
    \label{tab:treatments}
    \footnotesize
    \centering
    \begin{tabularx}{\textwidth}{|>{\hsize=0.3\hsize}X|>{\hsize=0.7\hsize}X|}
        \hline
        \multicolumn{2}{|c|}{Шифрлсэн файлын тайлах} \\
        \hline
        ID & 7 \\
        \hline
        Тодорхойлолт & \\
        \hline
        Өмнөх нөхцөл & \\
        \hline
        Үндсэн урсгал & \\
        \hline
        Дараах нөхцөл & \\
        \hline
      \end{tabularx}
\end{table}

%-------------------------------------------------------------------------------
%	SECTION 2
%-------------------------------------------------------------------------------
\section{Системийн загвар}

\subsection*{Үйл ажилгааны диаграмм}

\subsection*{Өгөгдлийн сангийн бүтэц}

%-------------------------------------------------------------------------------
%	SECTION 3
%-------------------------------------------------------------------------------
\section{Системийн хөгжүүлэх}

%-------------------------------------------------------------------------------
%	SECTION 4
%-------------------------------------------------------------------------------
\section{Файл хуваалцах системийг турших}
%-------------------------------------------------------------------------------
%	SECTION 5
%-------------------------------------------------------------------------------

\section{Дүгнэлт}


%\pagecolor{Beige}
% \chapter{Ерөнхий дүгнэлт}

\addchaptertocentry{Дүгнэлт}
\pagecolor{white}

%\begin{page}
\section*{\centering \huge{Дүгнэлт}}
Мэдээллийн эрин зуунд их өгөгдөл хуваалцах нь маш олон давуу талтай. Өгөгдөл хуваалцах олон ашиг тустай ч өгөгдлийн аюулгүй байдал өгөгдлийг аюулгүй хуваалцах чухал юм. 

Энэ дипломын ажлаар прокси дахин шифрлэх схемийг ашиглан аюулгүй өгөгдөл хуваалцах системийн туршилтын загварыг хөгжүүллээ.

%\end{page}


%-------------------------------------------------------------------------------
%	THESIS CONTENT - APPENDICES
%-------------------------------------------------------------------------------

\appendix % Дараах "chapters" нь Хавсралт болохыг LaTex -д хэлэх
% Тезисийн бүлгүүдийг Appendices хавтаснаас бие даасан файл байдлаар оруулах
% \pagecolor{beeige}
% % Хавсралт A

\chapter{Хавсралтын нэр} % Main appendix title

\label{AppendixA} % For referencing this appendix elsewhere, use \ref{AppendixA}
\pagecolor{white}



Хавсралтыг эндээс эхэлж бичнэ. 


% \pagecolor{beeige}
% \chapter{LED контроллер AT89C51ED2} % Main appendix title

\label{AppendixB} % For referencing this appendix elsewhere, use \ref{AppendixA}
\pagecolor{white}

%        ---------------Програмын кодыг бичсэн жишээ 
\begin{lstlisting}
  ;   2020-12-13  last ver
;----------------------------------------------------------   595 shift
PDAT         	EQU       P0      ;shift data
r_A		EQU	   P10
r_B		EQU	   P11
OE          	EQU        P12	  ;down 595 enable                                                               
SRCLK		EQU	  P35     ;595 latch serial data shift clock          CLK
RCLK		EQU	  P36     ;595 latch parallel data output clock       SCLK
PL_1		EQU	  P37	  ;select 1st reg    

COIL_RLY        EQU       P13     ;        
;----------------------------------------------------------
RxD_OK           equ      01h   ;
;==========================================================
Buff_6               EQU    30h                
Buff_5               EQU    31h 
Buff_4               EQU    32h
Buff_3               EQU    33h                
Buff_2               EQU    34h 
Buff_1               EQU    35h
;----------
Buff_01              EQU    36h 
Buff_00              EQU    37h
;----------------------------------  
Numb_6             EQU     38h    ; High
Numb_5             EQU     39h
Numb_4             EQU     3Ah
Numb_3             EQU     3Bh
Numb_2             EQU     3Ch
Numb_1             EQU     3Dh    ;Low
;------------
Numb_01             EQU    3Eh
Numb_00             EQU    3Fh    ; 

time_sec            EQU    40h
;-------------------------  ADR_DISPLAY RAM --------------
Disp_6             EQU     09h    ; High
Disp_5             EQU     08h
Disp_4             EQU     07h
Disp_3             EQU     06h
Disp_2             EQU     05h
Disp_1             EQU     04h    ;Low
;-------------
Disp_01            EQU     03h
Disp_00            EQU     02h    ;  
;----------------------------------------- 
Beg_adr       EQU    0002h       ;0000h      ; begin RAM adr shif

;****************************************************************
org 0000
ljmp START                  
org 0003h                             
reti        
org 000bh                                        ;TF0  clock
reti        ;ljmp CLOCK               
org 0013h
reti
org 001bh
reti	
org 0023h
ljmp SER_PORT               ; RI+TI serial intrupt from PC
;****************************************************************
START:            mov sp,#0c8h                    ;200 
mov scon,#01010100b  ; mode1,variable speed
mov tl1,#0fDh        ; (BAUDRATE=9600 IF OSC = 22.1184MHz)
mov th1,#0fDh	  ;	RELOAD BAUDRATE VALUE
;--TH1=0F7H (BAUDRATE=9600 IF OSC = 33.1776MHz)
;--TH1=0FAH (BAUDRATE=14400 IF OSC = 33.1776MHz Timer1 mode2 SMOD=0)

mov AUXR,#00010000b
mov AUXR1,#0000000b

mov tmod,#00100001b ; GA1=0, c/T1=0, mode1=10,     GA0=0,c/T0=1,  mode0=01
mov tcon,#01000000b ; TF1=0,  TR1=1, TF0=0, TR0=0, IE1=0, IT1=0, IE0=0, IT0=0 
mov IE,#00010010b  ; EA= 0, -=0,  ET2=0, ES= 1, ET1=0, EX1=0, ET0=1, EX0=0
clr  TR0
setb EA
;--------------------------------------------  default  
MOV PDAT,#0
clr r_A
clr r_B
clr OE                      ;setb OE
clr RCLK
clr SRCLK
setb PL_1
clr COIL_RLY 
;---------------------------------------------------------------         
mov dptr,#0000h                     ; Clear Disp RAM
mov a,#00h 
mov r2,#160                             
clr1:   movx @dptr,a
inc dptr
djnz r2,clr1 
;-----------------------------
clr RxD_OK 
mov time_sec,#00h

mov Numb_6,#20h        ; High   Disp_6 
mov Numb_5,#20h        ;        Disp_5
mov Numb_4,#20h        ;        Disp_4
mov Numb_3,#20h        ;        Disp_3
mov Numb_2,#20h        ;        Disp_2
mov Numb_1,#20h        ; Low    Disp_1

mov Numb_01,#20h       ; 
mov Numb_00,#2eh      ; $
;----------------------------- Copy Disp RAM ------------------
mov r6,#Disp_6      ; High    
mov a,Numb_6
Lcall chr_beg
mov r6,#Disp_5
mov a,Numb_5
Lcall chr_beg
mov r6,#Disp_4
mov a,Numb_4
Lcall chr_beg
mov r6,#Disp_3
mov a,Numb_3
Lcall chr_beg
mov r6,#Disp_2
mov a,Numb_2
Lcall chr_beg
mov r6,#Disp_1       ; Low
mov a,Numb_1
Lcall chr_beg
;----------------
mov r6,#Disp_01
mov a,Numb_01
Lcall chr_beg
mov r6,#Disp_00       
mov a,Numb_00
Lcall chr_beg
;***************************************************************** 
MAIN_00:  jbc RxD_OK,New_dat 
mn_01:  lcall REFRESH  
sjmp MAIN_00

New_dat: mov Numb_6,Buff_6 
mov Numb_5,Buff_5 
mov Numb_4,Buff_4 
mov Numb_3,Buff_3 
mov Numb_2,Buff_2
mov Numb_1,Buff_1  
;-----------------
mov Numb_01,Buff_01
mov Numb_00,Buff_00  

;----------------------------- Copy Disp RAM ------------------
mov r6,#Disp_6      ; High    
mov a,Numb_6
Lcall chr_beg
mov r6,#Disp_5
mov a,Numb_5
Lcall chr_beg
mov r6,#Disp_4
mov a,Numb_4
Lcall chr_beg
mov r6,#Disp_3
mov a,Numb_3
Lcall chr_beg
mov r6,#Disp_2
mov a,Numb_2
Lcall chr_beg
mov r6,#Disp_1       ; Low
mov a,Numb_1
Lcall chr_beg
;----------------
mov r6,#Disp_01
mov a,Numb_01
Lcall chr_beg
mov r6,#Disp_00       
mov a,Numb_00
Lcall chr_beg

Lcall CHECK_RELAY
Ljmp mn_01 
;================================================================
;                                SUB CALL  
;================================================================ 
chr_beg:          mov b,#83                    ;0c0h
mul ab
mov dpl,a
mov a,b
orl a,#80h
mov dph,a              ;adr dptr=  begin code 
inc dptr
;---------------------------------------------------------------
INC AUXR1
mov dpl,r6
mov dph,#00h
INC AUXR1                                ; (optional) restore DPS  
;----------------------- to RAM  shift ASCII code  2byte x 16 row  
mov b,#16                        ;num_row  ; row=16
asc_0:       mov a,#00h
movc a,@a+dptr
inc dptr
inc dptr
;------
INC AUXR1                        ;   switch data pointers
movx @dptr,a                     ; adr_beg 0002h RAM
clr c  
mov a,dpl
addc a,#10                       ;offset_shift =10-1=9
mov dpl,a
mov a,dph
addc a,#0
mov dph,a
INC AUXR1                        ; (optional) restore DPS  
;---------
djnz b,asc_0
ret
;--------------------------------------------------------------
clear_disp:    mov dptr,#0000h                  ;#Beg_adr                            
mov r2,#160
mov a,#00h            
rep_mem1:   movx @dptr,a
inc dptr
djnz r1,rep_mem1
ret


\end{lstlisting}
% \pagecolor{beeige}
% \chapter{Хавсралтын нэр} % Main appendix title

\label{AppendixC} % For referencing this appendix elsewhere, use \ref{AppendixA}
\pagecolor{white}

Хавсралтыг эндээс эхэлж бичнэ. 

%-------------------------------------------------------------------------------
%	BIBLIOGRAPHY
%-------------------------------------------------------------------------------
\defformat % Бүлгийн нэрийг оргиналь байдлаар хэвлэх

\addchaptertocentry{Ном зүй} % Ном зүйг гарчигт нэмэх

\printbibliography[heading=bibliography,title={Ном зүй}]

%-------------------------------------------------------------------------------
\end{document}  
 