%-------------------------------------------------------------------------------
%	WORK PLAN & REVIEW PAGE
%-------------------------------------------------------------------------------

%---------------------ТӨЛӨВЛӨГӨӨНИЙ ХУУДСЫН ЭХЛЭЛ--------------------------------
\begin{titlepage}

\noindent Батлав. \deptname ын эрхлэгч: 
\begin{flushright}
\makebox[6cm]{\dotfill} /\chairname/ 
\end{flushright} 
Удирдагч:\makebox[6cm]{ \dotfill} /\supname/
%\begin{flushright}
%\makebox[4cm]{ \dotfill} /\supname/
%\end{flushright}
\begin{center}
\vspace*{0.5cm}
\textbf{{\large \textsc{Дипломын төсөл гүйцэтгэх төлөвлөгөө}}}\\[0.5cm]
\end{center}
%\vspace*{0.5cm}
\noindent \textbf{Дипломын төслийн сэдэв:}\\
\textbf{Монгол}: '' \ttitle '' \\
\textbf{Англи}: '' \ttitleng ''\\

\noindent \textbf{Төслийн зорилго}: Proxy Re-Encryption схемийн хэрэглээнүүдийг судалж, нэгэн хэрэглээг хэрэгжүүлэх туршилтын систем хөгжүүлэх\\
 
\noindent \textbf{Гүйцэтгэх оюутны овог нэр}:\makebox[4cm]{ } \shortname /\studentcode/ \\
\textbf{Холбоо барих утас}: \makebox[7cm]{ }\phonenum \\
\noindent
	\begin{tabularx}{1\textwidth}{| >{\hsize=0.2\hsize}X
		| >{\hsize=2.8\hsize}Z
		| >{\hsize=0.5\hsize}Z
		| >{\hsize=0.5\hsize}Z |}
	\hline
	\multirow{2}{*}{№} &\multirow{2}{*}{Ажлын бүлэг, хэсгийн нэр} & \multirow{2}{4em}{\centering эзлэх хувь} & \multirow{2}{5em}{\centering дуусах хугацаа} \\ 
	& & & \\ \hline
	\multicolumn{4}{|l|}{\parbox[l]{13cm}{Бүлэг №1. Прокси дахин шифрлэлт схемийн онолын хэсэг}} \\  \hline
	\multirow{3}{*}{} &\multirow{3}{*}{} & \multirow{3}{*}{} & \multirow{3}{*}{} \\
	1 & \parbox[l]{9cm}{
		1.1 Шифрлэлт, түүний ач холбогдол, ангилал, хэрэглээ\\
		1.2 Өгөгдөлийг аюулгүй хуваалцах схемүүд \\
		1.3 Proxy Re-Encryption схем
		} & 20\% & \\ & & & \\ \hline
	\multicolumn{4}{|l|}{\parbox[l]{13cm}{Бүлэг №2. Прокси дахин шифрлэлт серверт шифрлэгдсэн файл хуваалцах судалгаа}} \\ \hline
	\multirow{3}{*}{} &\multirow{3}{*}{} & \multirow{3}{*}{} & \multirow{3}{*}{} \\
	2 & \parbox[l]{9cm}{
		2.1 Файл хуваалцах үйлчилгээнүүд \\
		2.1 Файл шифрлэх аргууд \\
		2.2 Клоуд орчинд өгөгдөл файл хадаглах
		} & 40\% &  \\  & & & \\ \hline
	\multicolumn{4}{|l|}{\parbox[l]{13cm}{Бүлэг №3. Proxy re-encryption систем хөгжүүлэх}} \\ \hline
	\multirow{2}{*}{} &\multirow{2}{*}{} & \multirow{2}{*}{} & \multirow{2}{*}{} \\
	3 & \parbox[l]{9cm}{
		3.1 Системийн үйл ажилгааны загвар\\
		3.2 Хөгжүүлэх технологи, хэл сонгох\\
		3.3 Системийн хөгжүүлэх
		} & 40\% & \\ & & & \\ \hline
	\multicolumn{4}{|l|}{Бүлэг №4. Ерөнхий дүгнэлт} \\  \hline
%	\multirow{2}{*}{} &\multirow{2}{*}{} & \multirow{2}{*}{} & \multirow{2}{*}{} \\
%4 & \parbox[l]{9cm}{
%	4.1 Нэмэлт бүлгүүдийг бичих\\
%	4.2 Хураангуй бичих \\
%	4.3 Дүгнэлт бичих \\
%	4.4 Дипломын бичвэрийг бүрэн боловсруулж дуусгах} & 20\% & V.25 \\  \hline
\end{tabularx}

\vspace{0.5cm}
Төлөвлөгөөг боловсруулсан оюутан: \makebox[3cm]{\dotfill} /\shortname/

%end{center}
\end{titlepage}
%---------------------ТӨЛӨВЛӨГӨӨНИЙ ХУУДСЫН ТӨГСГӨЛ ---------------------------------------------------

%---------------------ҮЗЛЭГИЙН ХУУДСЫН ЭХЛЭЛ ---------------------------------------------------
\newpage
\begin{titlepage}
	
	\small	
	\begin{center}
		\textbf{{\large ТӨГСӨЛТИЙН АЖЛЫН ҮЗЛЭГИЙН ХУУДАС}}\\
	\end{center}
	\vspace*{0.5cm}
	\noindent Оюутны код: \studentcode \\
	Оюутны нэр: \shortname \\
	Сэдвийн монгол нэр: '' \ttitle '' \\
	Сэдвийн англи нэр: '' \ttitleng ''\\
	Удирдагч багш: \supname\\
	Зөвлөгч багш: \advicenameA, \advicenameB \\
	
	\noindent	\begin{tcolorbox}[tab2,tabularx={ >{\hsize=0.2\hsize}Z| 		
			>{\hsize=0.8\hsize}Z |
			>{\hsize=1.0\hsize}Z|
			>{\hsize=0.9\hsize}Z|
			>{\hsize=2.1\hsize}Z
		},boxrule=0.9pt]
		№ & Үзлэгийн гүйцэтгэл & Гүйцэтгэлийн 30\% -с багагүй байна. & Огноо & Удирдагч \supname \hspace{0.1cm} багшийн гарын үсэг \\ \hline
		\multirow{3}{*}{1} & \multirow{3}{*}{Үзлэг-1} &    & \multirow{3}{*}{IV/03-IV/07} &  \\
		& & & & \\
		& & & & 
	\end{tcolorbox}
	Багшийн товч зөвлөгөө, тайлбар:
	\begin{center}
		\dotfill \\ [0.1cm]
		\dotfill \\ [0.1cm]
		\dotfill \\ [0.1cm]
		\dotfill \\ [0.1cm]
		\dotfill \\ [0.1cm]
		\dotfill \\ [0.1cm]
		\dotfill \\ [0.1cm]	
		\vspace{0.2cm}
		Үзлэг-1 хийсэн багш:\makebox[3cm]{\dotfill} /\supname/
	\end{center}
	\vspace{1cm}
	\noindent	\begin{tcolorbox}[tab2,tabularx={ >{\hsize=0.2\hsize}Z| 		
			>{\hsize=0.8\hsize}Z |
			>{\hsize=0.9\hsize}Z|
			>{\hsize=1.2\hsize}Z|
			>{\hsize=0.9\hsize}Z|
			>{\hsize=2.0\hsize}Z
		},boxrule=0.9pt]
		№ & Үзлэгийн гүйцэтгэл &Авсан оноо (10 оноо) &Гүйцэтгэлийн 50\% -с багагүй байна. & Огноо & \advicenameA \hspace{0.1cm} багшийн гарын үсэг \\ \hline
		\multirow{3}{*}{1} & \multirow{3}{*}{Үзлэг-2} &  &  & \multirow{3}{*}{IV/17-IV/21} &  \\
		& & & & & \\
		& & & & &
	\end{tcolorbox}
	Багшийн товч зөвлөгөө, тайлбар:
	\begin{center}
		\dotfill \\ [0.1cm]
		\dotfill \\ [0.1cm]
		\dotfill \\ [0.1cm]
		\dotfill \\ [0.1cm]
		\dotfill \\ [0.1cm]
		\dotfill \\ [0.1cm]
		\dotfill \\ [0.1cm]	
		\vspace{0.2cm}
		Үзлэг-2 хийсэн багш:\makebox[3cm]{\dotfill} /\advicenameA/
	\end{center}
\end{titlepage}
\newpage
\begin{titlepage}
\small	
	\begin{center}
	\textbf{{\large ТӨГСӨЛТИЙН АЖЛЫН ҮЗЛЭГИЙН ХУУДАС}}\\
\end{center}
\vspace*{0.5cm}
\noindent Оюутны код: \studentcode \\
Оюутны нэр: \shortname \\
Сэдвийн монгол нэр: '' \ttitle '' \\
Сэдвийн англи нэр: '' \ttitleng ''\\
Удирдагч багш: \supname\\
Зөвлөгч багш: \advicenameA, \advicenameB \\
\noindent	\begin{tcolorbox}[tab2,tabularx={ >{\hsize=0.2\hsize}Z| 		
		>{\hsize=0.8\hsize}Z |
		>{\hsize=0.9\hsize}Z|
		>{\hsize=1.2\hsize}Z|
		>{\hsize=0.9\hsize}Z|
		>{\hsize=2.0\hsize}Z
	},boxrule=0.9pt]
	№ & Үзлэгийн гүйцэтгэл &Авсан оноо (10 оноо) &Гүйцэтгэлийн 70\% -с багагүй байна. & Огноо & \advicenameB \hspace{0.1cm} багшийн гарын үсэг \\ \hline
	\multirow{3}{*}{1} & \multirow{3}{*}{Үзлэг-3} &  &  & \multirow{3}{*}{V/08-V/12} &  \\
	& & & & & \\
	& & & & &
\end{tcolorbox}
Багшийн товч зөвлөгөө, тайлбар:
\begin{center}
	\dotfill \\ [0.1cm]
	\dotfill \\ [0.1cm]
	\dotfill \\ [0.1cm]
	\dotfill \\ [0.1cm]
	\dotfill \\ [0.1cm]
	\dotfill \\ [0.1cm]
	\dotfill \\ [0.1cm]	
	\vspace{0.2cm}
	Үзлэг-3 хийсэн багш:\makebox[3cm]{\dotfill} /\advicenameB/
\end{center}		
\vspace{1cm}
\noindent	\begin{tcolorbox}[tab2,tabularx={ >{\hsize=0.2\hsize}Z| 		
		>{\hsize=0.8\hsize}Z |
		>{\hsize=1.0\hsize}Z|
		>{\hsize=0.9\hsize}Z|
		>{\hsize=2.1\hsize}Z
	},boxrule=0.9pt]
	№ & Үзлэгийн гүйцэтгэл & Гүйцэтгэлийн 90\% -с багагүй байна. & Огноо & Удирдагч \supname \hspace{0.1cm} багшийн гарын үсэг \\ \hline
	\multirow{3}{*}{1} & \multirow{3}{*}{Үзлэг-4} &    & \multirow{3}{*}{V/15-V/19} &  \\
	& & & & \\
	& & & & 
\end{tcolorbox}
\vspace{0.5cm}
\noindent	\begin{tcolorbox}[tab2,tabularx={ >{\hsize=0.2\hsize}Z| 		
		>{\hsize=1.6\hsize}Z |
		>{\hsize=0.8\hsize}Z|
		>{\hsize=1.4\hsize}Z
	},boxrule=0.9pt]
	№ & Удирдагч \supname \hspace{0.1cm} багшийн үнэлгээ (30 оноо) & Огноо & Удирдагч багшийн гарын үсэг \\ \hline
	\multirow{3}{*}{1} &    & \multirow{3}{*}{V/17} &  \\
	& & & \\
	& & & 
\end{tcolorbox}
\begin{center}
\vspace{0.5cm}
	Удирдагч  багш:\makebox[3cm]{\dotfill} /\supname/ \\[0.5 cm]
	\textit{\footnotesize Жич: Удирдагч багш өөрийн үнэлгээгээ 30 хүртэл оноогоор өгөх ба үнэлгээ тавьсан хуудсыг оюутанд буцааж өгөлгүй төгсөлтийн нарийн бичгийн даргад хураалгана уу.}
\end{center}
\end{titlepage}

%---------------------ҮЗЛЭГИЙН ХУУДСЫН ТӨГСГӨЛ ---------------------------------------------------



%---------------------ГҮЙЦЭТГЭЛИЙН ХУУДСЫН ЭХЛЭЛ ---------------------------------------------------
\newpage

\begin{titlepage}

\begin{center}

\vspace*{2cm}
\textbf{{\large ТӨГСӨЛТИЙН АЖЛЫН ЯВЦ}}\\[0.5cm]

\begin{tabularx}{1\textwidth}{| >{\hsize=0.1\hsize}Z
		| >{\hsize=2.5\hsize}X
		| >{\hsize=0.6\hsize}Z
		| >{\hsize=0.8\hsize}Z |}
	\hline
	\multirow{2}{*}{№} & \multirow{2}{*}{Хийж гүйцэтгэсэн ажил} & Биелсэн & Удирдагчийн \\
	  & & хугацаа & гарын үсэг \\ \hline
	1 & {Бүлэг №1. Proxy Re-Encryption схемийн онолын хэсэг} & 2023-4-28 &  \\ \hline
	2 & {Бүлэг №2. Серверт шифрлэгдсэн файл хуваалцах судалгаа} & 2023-4-21 &  \\ \hline
	3 & {Бүлэг №3. Proxy re-encryption систем хөгжүүлэх} & 2023-5-18 &  \\ \hline
	4 & {Бүлэг №4. Ерөнхий дүгнэлт}    & 2023-5-25 &  \\ \hline
\end{tabularx}

\vspace{1cm}
Ажлын товч дүгнэлт \\[0.5cm]

\dotfill \\ [0.2cm] 
\dotfill \\ [0.2cm]
\dotfill \\ [0.2cm]
\dotfill \\ [0.2cm]
\dotfill \\ [0.2cm]
\dotfill \\ [0.2cm]
\dotfill \\ [0.5cm]

Удирдагч: \makebox[3cm]{\dotfill} /\supname/ \\

\vspace{2cm}
ЗӨВШӨӨРӨЛ \\[0.5cm]
Оюутан \shortname --н бичсэн төгсөлтийн ажлыг УШК-д хамгаалуулахаар тодорхойлов.\\[0.5cm]
Салбарын эрхлэгч: \makebox[3cm]{\dotfill} /\chairname/
\end{center}

\end{titlepage}
%---------------------ГҮЙЦЭТГЭЛИЙН ХУУДСЫН ТӨГСГӨЛ ---------------------------------------------------

%\begin{} ---------------------ШҮҮМЖИЙН ХУУДСЫН ЭХЛЭЛ ---------------------------------------------------
\newpage

\begin{titlepage}
\begin{center}

{\scshape\Large \univname\par} % Их сургуулийн нэр
{\scshape\large \facname\par}\vspace{1cm} % Их сургуулийн нэр

\textbf{{\Large ШҮҮМЖИЙН ХУУДАС}}\\[1cm]

\end{center}

\normalsize

\deptname --н салбарын төгсөх курсийн оюутан \shortname -н ''\ttitle'' сэдэвт төгсөлтийн ажлын шүүмж.

\begin{enumerate}
\item Төслөөр дэвшүүлсэн асуудал, үүнтэй холбоотой онолын материал уншиж судалсан байдал. Энэ талаар хүмүүсийн хийсэн судалгаа, түүний үр дүнг уншиж тусгасан эсэх.
\begin{center}
\dotfill \\[0.1cm]
\dotfill \\[0.1cm]
\dotfill \\[0.1cm]
\dotfill \\[0.1cm]
\dotfill \\[0.1cm]
\dotfill \\[0.1cm]
\dotfill \\[0.4cm]
\end{center}
\item Төслийн ерөнхий агуулга. Шийдсэн зүйлүүд, хүрсэн үр дүн. Өөрийн санааг гарган, харьцуулалт хийн, дүгнэж байгаа чадвар.
\begin{center}
\dotfill \\[0.1cm]
\dotfill \\[0.1cm]
\dotfill \\[0.1cm]
\dotfill \\[0.1cm]
\dotfill \\[0.1cm]
\dotfill \\[0.1cm]
\dotfill \\[0.4cm]
\end{center}
\item Эмх цэгцтэй, стандарт хангасан өөрөөр хэлбэл диплом бичих шаардлагуудыг биелүүлсэн эсэх. Төсөлд анзаарагдсан алдаанууд, зөв бичгийн болон өгүүлбэр зүйн гэх мэт /Хуудас дугаарлагдаагүй, зураг хүснэгтийн дугаар болон тайлбар байхгүй, шрифт хольсон, хувилсан зүйл ихээр оруулсан/.
\begin{center}
\dotfill \\[0.1cm]
\dotfill \\[0.1cm]
\dotfill \\[0.1cm]
\dotfill \\[0.1cm]
\dotfill \\[0.1cm]
\end{center}
\end{enumerate}
\end{titlepage}

\newpage

\begin{titlepage}
\begin{enumerate}
\item[4.] Төслөөр орхигдуулсан болон дутуу болсон зүйлүүд. Цаашид анхаарах хэрэгтэй зүйлүүд.
\begin{center}
\dotfill \\[0.1cm]
\dotfill \\[0.1cm]
\dotfill \\[0.1cm]
\dotfill \\[0.1cm]
\dotfill \\[0.1cm]
\dotfill \\[0.1cm]
\dotfill \\[0.4cm]
\end{center}
\item [5.] Төслийн талаар онцолж тэмдэглэх зүйлүүд.
\begin{center}
\dotfill \\[0.1cm]
\dotfill \\[0.1cm]
\dotfill \\[0.1cm]
\dotfill \\[0.1cm]
\dotfill \\[0.1cm]
\dotfill \\[0.1cm]
\dotfill \\[0.4cm]
\end{center}
\item [6.] Ерөнхий оноо. (30 оноо)
\begin{center}
\dotfill \\[1cm]
\end{center}
\end{enumerate}
Шүүмж бичсэн: \makebox[3cm]{\dotfill} /\readname/ \\[0.5cm]
Ажлын газар: \dotfill \\[0.5cm]
Хаяг (Утас) \makebox[5cm]{\dotfill}
\end{titlepage}
%\end{} ---------------------ШҮҮМЖИЙН ХУУДСЫН ТӨГСГӨЛ ---------------------------------------------------