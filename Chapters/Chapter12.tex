% Бүлэг 2

\chapter{Онолын хэсэг} % Зарим нэг зөвлөмж

\label{Chapter2} % Энэ бүлэг рүү ишлэл хийх бол \ref{Chapter2} командыг ашигла 
%-------------------------------------------------------------------------------
%	SECTION 1
%-------------------------------------------------------------------------------
\pagecolor{white}
\section{Програмчлагддаг логик контроллер (PLC)}
Програмчлагддаг логик контроллер (PLC) гэдэг нь цогц системийг удирдахад үр ашигтай, уян хатан өөрөөр хэлбэл өөр төрлийн системийг хурдан бөгөөд хялбар удирдах боломжтой тооцоолох чадвар нь нарийн удирдахад өргөн хэрэглэдэг. Алдаагаа илрүүлэх боломж нь програмчлалыг хялбар болгож өгдөг. Найдвартай эд ангиуд нь олон жилийн турш ажиллах боломжтой.\cite{PLC} \\
\textbf{CPU} - дотоод төв процессороор микро процессор ашигладаг. \\
\textbf{RAM} - програмын өгөгдлийг хадгалдаг. \\
\textbf{ROM} - үйлдлийн систем, драйвер болон хэрэглээний програмыг хадгалдаг.\\
\textbf{Оролт, гаралт (input, output)} - энд процессор гадаад төхөөрөмжөөс мэдээлэл авах болон мэдээлэл дамжуулна. \\
\textbf{PSU} – Хучдэлийн түвшин нь PLC-ээсээ хамаараад 24vdc,120vac,220vac байна.
\begin{figure}[!ht]
	\centering
	\includegraphics[scale=0.15]{2-1}
	\caption{PLC SIEMENS S7-224 XP CPU AC/DC/RLY}
	\label{fig:2-1}
\end{figure}
\subsection{ПЛК програмчлал}
ПЛК програм нь текст болон график хэлбэрээс бүрдэх бөгөөд энэ нь ПЛК-ыг хянаж буй процессыг зохицуулдаг логикийг илэрхийлдэг. ПЛК програмчлалын хэлний хоёр үндсэн ангилал байдаг бөгөөд дээрх хоёр ангиллыг цаашид олон дэд ангилалд хуваадаг. 
\begin{enumerate}
	\item Текст хэлбэр 
	\begin{itemize}
		\item Зааварчилгааны жагсаалт 
		\item Зохион байгуулалттай текст 
	\end{itemize}
\item График хэлбэр 
\begin{itemize}
	\item 	Шаталсан диаграм (LD)
	\item Функцийн блок диаграм (FBD)
	\item Дараалсан функцийн диаграм (SFC)
\end{itemize}
\end{enumerate}
Энгийн бөгөөд ашиглахад хялбар байдлаас шалтгаалан график хэлбэрийн програмчлалыг текст хэлбэрийн програмчлалаас илүү түлхүү ашигладаг. 
Функцийн блок диаграм нь ПЛК-д олон функцийг хялбар програмчлах боломжийг олгодог. Функцийн блокийн онцлог нь хэдэн ч тооны оролт, гаралтыг холбож болох бөгөөд нэг функцийн блокийн гаралтыг өөр нэг функцийн блокийн оролтод холбож болно. Ингэж холбож явсаар функцийн блок диаграм болон өргөждөг.

\subsubsection{Шаталсан логик (Ladder logic) }
Шаталсан логик нь ПЛК програмчлалын хамгийн энгийн хэлбэр юм. Релэйний хяналттай системүүдийн реле контактуудыг удирдахад шаталсан логикийг ашигладаг. Ladder logic нь PLC –д хэрэглэгддэг үндсэн програмчлалын арга юм. Ladder logic нь релейн логикийг дууриан хөгжиж ирсэн. Ladder logic оролтууд normal open, normal closed, immediate inputs гэсэн гурван төрөл байна. \cite{EE343}\\

 Ladder logic гаралтууд нь: 
Эхнийх нь энгийн гаралт ташуу зураастай тойрог гаралт нь эхний гаралтын урвуу one shot relay эхний шалгалтын үед нэг асаах ба дараа нь унтарч дараагийн шалгалтын унтарсан байх бөгөөд уг заагчийг унтраатал үргэлжлэх болно. L болон U заагч нь гаралтыг түгжихэд хэрэглэгддэг. L заалт нь энергижих үед гаралт асах ба U гаралтын тусламжтай унтраана. Immediate output гаралт нь Ladder logic-н гаралтыг хүлээхгүйгээр асуудлыг шийддэг.

\section{Лед дэлгэц}
Гадна орчинд ажиллах LED матрицан дэлгэц (Red color LED modul  Pixel 32x16)
\begin{figure}[!ht]
	\centering
	\begin{subfigure}[b]{0.3\textwidth}
		\includegraphics[width=\textwidth]{2-2a}
		\caption{нүүрэн тал}
	\end{subfigure}
	~
	\begin{subfigure}[b]{0.3\textwidth}
		\includegraphics[width=\textwidth]{2-2b}
		\caption{холболт}
	\end{subfigure}
	~
	\begin{subfigure}[b]{0.3\textwidth}
		\includegraphics[width=0.5\textwidth]{2-2c}
		\caption{портын сигнал}
	\end{subfigure}
	\caption[P10 - Outdoor LED Display Panel]{P10 - Outdoor LED Display Panel - 32x16 - High Brightness RED - 5V - Dot Matrix Display}
\end{figure}

\begin{tabular}{|c|c|}
	\hline
	 Загвар &	 P10 \\ \hline
	1 модуль дахь цэгийн тоо &	32*16 pixel    =     512 цэг  \\ \hline
	Цэгийн өнгө  &	 улаан   \\ \hline           
	Долгионы урт  &	620-650  nm \\ \hline
	Цэгийн нягтрал &	 10000 dot/ m*m \\ \hline
	Модулийн хэмжээ&	 320mm * 160mm \\ \hline
	Скан хийх горим	& 1/4  scan \\ \hline
	Усанд тэсвэртэй эсэх &	гадаа зориулалтын усанд тэсвэртэй \\ \hline
	 Гэрэлтэлтийн хүч	&  1500cd/ m2 -аас их \\ \hline
	Харагдах өнцөг	& Хэвтээ/босоо тэнхлэгт: 80/60 градус \\ \hline
	Ажиллах хугацаа & 100000 цагаас их \\ \hline
\end{tabular}

\subsection{Гэрлэн диод}
LED гэдэг нь гэрэл цацруулах диод гэсэн үгний товчлол юм. LED-ийн гэрэлтэх технологийн үндэс нь зарим хагас дамжуулагч материалууд цахилгаан зарцуулалтын түвшин болон материалын төрлөөс хамааран тусгайлсан долгионы уртыг гарган гэрэлтэх явцыг бий болгодог бөгөөд тод бүдэг болон харагдах өнцөг нь өөр өөр байж болдог. Хагас дамжуулагч гэрэлт диодын дэлгэцийн горимоос хамааран дэлгэцэнд текст, график, зураг, видео, хөдөлгөөнт дүрс, видео сигнал болон бусад мэдээллийг харуулдаг.
\subsection{Зүүмэл LED дэлгэцүүд}
Зүүмэл LED дэлгэцүүд нь доорх ангиллуудтай байдаг:
\begin{itemize}
	\item Хоёр өнгөтэй гадна байрладаг LED дэлгэцийг гол төлөв гудамж талбай, худалдааны төв, болон бусад хүн олон цугладаг гаднах талбайд байрлуулдаг.                                            
	\item Бүрэн өнгөтэй дотор байрладаг LED дэлгэц буюу гурван үндсэн өнгөтэй хэмээн нэрлэдэг дэлгэцийг жижиг хэмжээ шаардсан дэлгэцүүдэд хэрэглэдэг. 
	\item Бүрэн өнгөтэй гадна байрладаг LED дэлгэц буюу мөн дээрхийн адил гурван үндсэн өнгөтэй хэмээн нэрлэдэг дэлгэцийг жижиг хэмжээ шаардсан дэлгэцүүдэд хэрэглэдэг. 
	\item Хоёр өнгөтэй дотор байрладаг LED дэлгэцийг гол төлөв санхүүгийн байгууллага, шуудан, харилцаа холбоо, эмнэлэг, цэрэг арми, дэлгүүр, татварын алба болон бусад салбаруудад хэрэглэдэг. 
\end{itemize}

\subsection{LED дэлгэцийн үндсэн бүрдэл хэсгүүд}
LED дэлгэц нь зүүмэл дэлгэцүүдээс бүрддэг бөгөөд тохирох удирдлагын хэсэгтэй холбогдсон байдаг. Шаардлагатай үед төрөл бүрийн дэлгэцийн хавтангуудыг төрөл бүрийн ялгаатай удирдлагын хэсэгтэй холбон олон төрөлт LED дэлгэцийг бий болгож болно.
\subsection{LED дэлгэцийн нийтлэг параметрүүд}
\subsubsection{Пикселийн хэмжээ } 
Зэргэлдээ пикселүүдийн төвийн хоорондох зай (нэгж:мм)-г пикселийн хэмжээ гэнэ. 
\begin{figure}[!ht]
	\centering
	\includegraphics[scale=0.6]{2-4}
	\caption{Пикселийн хэмжээ}
	\label{fig:2-2}
\end{figure}
\subsubsection{Нягтрал }
Нэгж хэсэгт хамаарах пикселийн тоо хэмжээг (нэгж:цэг/$m^2$) нягтрал гэнэ. 
Цэгүүд болон пикселийн хэмжээ нь дараах хамааралтай. 
\[\textit{Нягтрал}=1000\div \textit{пикселийн хэмжээ}\] 
Өндөр нягтралтай LED дэлгэцүүд нь дүрсийг тод гаргадаг боловч сайн харагдах зай багатай байдаг.
\subsubsection{Өнцгийн түвшин}
Дэлгэцийн гадаргууг LED, пиксел, дэлгэцийн модуль бүрдүүлдэг гэж үзвэл түүний хотгор гүдгэрийн алдааг өнцгийн түвшин гэдэг. Өнцгийн түвшин нь муу LED дэлгэцийн хувьд өнгө гаргалт нь жигд бус байна.
\subsubsection{Гэрэлтэцийн түвшин}
Энэ нь харанхуйгаас гэрэлтэй рүү ижил төвшингөөр ялгарах гэрэлтэцийг хэлнэ. Гэрэлтэц нь мөн өнгөний ялгарал эсвэл гэрлийн харлалтын түвшин гэж нэрлэгддэг. \\
Тоон технологи дээр суурилсан дэлгэцийн хувьд энэ нь дэлгэцийн өнгөний тоог тодорхойлогч болдог. Гэрэлтэцийн түвшин нь системийн аналог тоон хувиргуурын битийн тооноос хамаардаг. Гэхдээ системийн видео боловсруулж буй схем, санах ой болон сигналыг дамжуулагч систем нь тохирсон зүйлээр хангах ёстой. Ерөнхийдөө гэрэлтэцийн түвшин нь 8, 16, 32, 64, 128, 256 гэх мэт байдаг. Гэрэлтэцийн түвшин өндөр байх нь тод, дан ганц өнгийг гаргаж чаддаг. \\
\begin{figure}[!ht]
	\centering
	\includegraphics[scale=0.6]{2-4a}
	
\end{figure}
Одоогоор бид 8 битийн LED дэлгэцийн боловсруулалтын системийг ашиглаж байгаа ба энэ нь 256 (28) өнгөний ялгаралтай. Үүнийг энгийнээр ойлгуулахад хараас цагаан руу 256 янзын гэрэлтэцийг бий болгоно гэсэн үг. Хэрэв үндсэн өнгө болох улаан, ногоон ба цэнхэр өнгө бүрийн 256 янзаар гэрэлтэцийг өөрчилбөл нийт 256 × 256 × 256 = 16777216 өнгө буюу 16 сая өнгийг гарган авч чадна. Гэхдээ олон улсын компаниуд голдуу 10 битийн дэлгэцийн боловсруулалтын системийг ашиглан 1024 түвшинг гаргадаг ба энэ нь RGB2-ээр илэрхийлбэл 1.07 миллиард өнгийг гаргана гэсэн үг.
\subsubsection{Фреймийн давтамж (фреймийн сэргээх давтамж)}
LED дэлгэцэнд гарч буй мэдээллийг сэргээх хугацааны хэмжээг фреймийн сэргээх давтамж гэнэ.
Гол төлөв 25Гц,120Гц, 240Гц болон дээш фреймийн давтамжтай байх нь дүрсийг тасралтгүй үргэлжлүүлэх чанар сайтай байдаг.
\subsubsection{Сэргээх хугацаа (сэргээх давтамж)}
LED дэлгэц нь нэг секундэд өгөгдлийн хэд хэдэн удаа давтан үзүүлдэг.
Ерөнхийдөө 60Гц, 120Гц, 240Гц болон түүнээс дээш сэргээх давтамжтай байх нь дүрсийн тогтвортой байдлыг илүү сайн хангадаг.
\subsection{LED дэлгэцийн систем}
LED дэлгэцийн систем нь үндсэн гурван хэсгээс бүрддэг. Үүнд: Сигналын үүсвэр, удирдлагын систем болон дэлгэцийн хэсэг багтдаг.
\textbf{Удирдлагын систем}: сигналын хандалт, дамжуулалт, боловсруулалт, хөрвүүлэлтийг хийх үндсэн үүрэгтэй.
\begin{figure}[!ht]
	\centering
	\includegraphics[scale=0.6]{2-5}
	\end{figure}
\begin{figure}[!ht]
	\centering
	\includegraphics[scale=0.8]{2-6}
	\end{figure}
\begin{figure}[!ht]
	\centering
	\includegraphics[scale=0.7]{2-7}
	\caption{At89c51ed2 microcontroller}
\end{figure}

