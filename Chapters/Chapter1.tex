% Бүлэг 1

\chapter{Sunil Patel --н зөвлөмж} % Бүлгийн нэр
\label{Chapter1} % Энэ бүлэг рүү ишлэл хийх бол \ref{Chapter1} командыг ашигла 
\pagecolor{white}
%-------------------------------------------------------------------------------

% Агуулгад ашигласан хэвшүүлэлтийн зарим командын тодорхойлолт
\newcommand{\keyword}[1]{\textbf{#1}}
\newcommand{\tabhead}[1]{\textbf{#1}}
\newcommand{\code}[1]{\texttt{#1}}
\newcommand{\file}[1]{\texttt{\bfseries#1}}
\newcommand{\option}[1]{\texttt{\itshape#1}}

%-------------------------------------------------------------------------------

\section{Тавтай морилно уу}
\LaTeX{} ашиглаж тезис бичих энэхүү гоёмсог, ашиглахад хялбарт загварт тавтай морилно уу. 

Хэрэв таны бичиж байгаа (бичихээр төлөвлөсөн) тезис техникийн эсвэл математикийн чиглэлээр бол \LaTeX{} ашиглахыг зөвлөж байна. Учир нь текст процессор дээр загвараа гаргах гэж цаг алдалгүй зөвхөн бичих зүйлдээ анхаарахад л болно.

\LaTeX{} бол хэдэн зуу, мянган хуудастай баримт бичгийг мэргэжлийн түвшинд хялбархан гаргаж чаддаг. Энгийн командын тусламжтай гарчиг, хуудасны зах, толгой болон хөлийг автоматаар үүсгэж форматын нэгдмэл байдал, үзэмжийг хангадаг. Түүний нэг гол хүч чадал нь \emph{хүнд} математикийг ч амархан бичиж чадна.

%-------------------------------------------------------------------------------

\section{\LaTeX{} --г сурах}

\LaTeX{} бол Microsoft Word, Adobe Pages шиг \textsc{WYSIWYG} (Таны Харж байгаа Зүйл бол Таны Оруулсан Зүйл) төрлийн текст боловсруулах програм биш. \LaTeX{} --д зориулсан баримт нь үнэндээ \emph{хэвшүүлээгүй} задгай текст бүхий файл юм. Өөрийн тексттэй хамт түүнийг хэрхэн хэвшүүлэхийг тухай энгийн командыг бичиж, энэ файлаар \LaTeX{} --т хэлж өгдөг. Жишээ нь \emph{текстийг налуу болгож тодотгохдоо} \verb|\emph{text}| командыг ашиглах ба налуу болгох текстээ их хаалтанд бичнэ. Өөрөөр хэлбэл \LaTeX{} нь HTML -тэй маш төстэй ''{mark-up}'' хэл юм.

\subsection{\LaTeX{} --н (тийм ч товч бус) танилцуулга}

{\LaTeX{}} таны хувьд шинэ зүйл бол маш сайн PDF цахим ном байдаг. Энэ бол \enquote{The Not So Short Introduction to \LaTeX{}}. Түүний сүүлийн хувилбарыг дараах сайтаас татаж болно:

{\url{http://www.ctan.org/tex-archive/info/lshort/english/lshort.pdf}}

Энэ ном бусад хэл дээр, түүний дотор монгол хэл дээр ч бий:\url{http://www.ctan.org/tex-archive/info/lshort/}

Танд \LaTeX{} --г ашиглаж сурахын тулд \url{http://www.LaTeXTemplates.com} сайт дээр байгаа зарим нэг загварыг ашиглаад энгийн 'тест' баримт үүсгэхийг зөвлөж байна. \emph{Үнэхээр} суръя гэвэл та тезисээ бичээд үз.

%-------------------------------------------------------------------------------

\section{Загварыг ашиглацгаая}

Хэрэв та \LaTeX{} --тай танилцсан бол загварын хавтаснаас \file{main.tex} файлыг нээж, өөрийн мэдээллийг оруулах \emph{THESIS INFORMATION} блокыг олоорой. Энд өөрийн сургууль болон боловсролын зэргийн талаарх өгөгдлөө оруулна. Яаж оруулахыг \ref{FillingFile} сэдэв (хуудас \pageref{FillingFile}) зааж өгнө. Энэ загварын ихэнх хэсэгт ашигласан тезистэй холбоотой зүйлсийг сэдэв \ref{ThesisConventions} --с уншина уу.

\LaTeX{} таны хувьд шинэ зүйл бол энэ баримтыг бүхэлд нь уншихыг зөвлөж байна.

Энэ загварыг ашиглахын өмнө түүний хэв шинж танай сургуулийн тезис бичих заавартай нийцэж байгааг шалгаарай. Ихэнх тохиолдолд нийцнэ гэж бодож байна. Танай сургуулийн зөвлөмжтэй нийцүүлэхэд бага зэргийн өөрчлөлт л шаардагдана. Ийм өөрчлөлтийг \file{MUSTThesis.cls} файлд хийх хэрэгтэй.

\subsection{Загварын тухай}

Тезисийн энэ \LaTeX{} загвар Их британи Өмнөд Тамптон их сургуулийн Электроник, компьютерийн ухааны профессор Стив Ганы загвар дээр суурилсан. Эх загварыг 
\url{http://www.ecs.soton.ac.uk/~srg/softwaretools/document/templates/} сайтаас авч болно.

Стив Ганы \file{ecsthesis.cls} --г Сунил Петер авч засаад хавтасны бүтэцтэй болгосон. Түүнийг \url{http://www.sunilpatel.co.uk/thesis-template} сайтаас харж болно. Энэ загвар хэрэглэгчдийн хүсэлтээр олон дахин өөрчлөгдсөн бөгөөд \url{http://www.LaTeXTemplates.com} нийтэд хүрч байна.

%-------------------------------------------------------------------------------

\section{Загварт юу багтсан бэ?}

\subsection{Хавтсууд}

Загварын zip --г зарлавал хэд хэдэн хавтаст орсон файлууд болно. Хавтасны нэр өөрөө дотор нь юу байгааг хэлнэ:

\keyword{Appendices} -- хавсралтын хавтас. Хавсралт бүр бие даасан \file{.tex} файл байна. Жишээ файл хавтаст бий.

\keyword{Chapters} -- тезисийн бүлгийн хавтас. Хавсралт бүр бие даасан \file{.tex} файл байх бөгөөд дараах байдлаар задарч болно:
\begin{itemize}
\item Бүлэг 1: Удиртгал
\item Бүлэг 2: Онолын судалгаа
\item Бүлэг 3: Төслийн хэсэг
\item Бүлэг 4: Дүгнэлт
\end{itemize}

\keyword{Figures} -- зургийн хавтас. Тезист орох зургийн файл энд байршина.

\keyword{FrontBackMatter} -- дагалдах материалын хавтас. Энэ хавтаст тезисийн нүүр хуудас, хураангуй, ашигласан ном зүй, тогтмол, таних тэмдэг зэрэг материалын файл энд байршина. Энэ бүх дагалдах материал тезис бүрд орох албагүй. Тезисийн төрөл, сургуулийн шаардлагаар заримыг нь ашиглаж болно. 

\subsection{Файлууд}

Энд агуулагдссан ихэнх файл задгай текст бөгөөд агуулгыг нь текст засварлагч дээр харж болно. Хөрвүүлэлт хийгдсэний дараа \LaTeX{}, BibTeX зэрэг програм завсрын түр зуурын файлууд үүсгэдэг. Тэднийг устгах эсэх талаар санаа зовох хэрэггүй.

\keyword{example.bib} -- энэ бол ном зүйн Bibtex ашиглаж ишлэл хийх бүхий л мэдээллийг агуулдаг чухал файл. Түүнийг гараар үүсгэж болох ч таны өмнөөс үүсгэж, удирдах програмууд байдаг. \LaTeX{} дахь ном зүй бол маш том сэдэв тул ашиглахын өмнө BibTeX --н талаар унших хэрэгтэй. 

\keyword{MUSTThesis.cls} -- энэ бол маш чухал файл. Энэ бол тезисийг яаж хэвшүүлэхийг тодорхойлсон класс файл юм.. 

\keyword{main.pdf} -- энэ бол \LaTeX{} --н бүтээсэн таны гоёмсог тезисийн PDF файл. Загвар - тезисийг хөрвүүлэхэд энэ файл үүснэ.

\keyword{main.tex} -- энэ бол маш чухал файл. Энэ файл \LaTeX{} -т таны тезисийг яаж хөрвүүлж PDF гаргаж авахыг хэлж өгдөг. Тэрээр тезисийн хэв шинжийн тогтолцоо, бүтээлтийг агуулдаг. Маш сайн тайлбарласан тул мөр бүр юу хийж байгааг уншаад ойлгоно. Өөрийн мэдээллийг \emph{THESIS INFORMATION} блокт оруулснаар та тезисээ бичиж эхэлнэ!

Загварт \emph{ороогүй} ч \LaTeX{} -ийн үүсгэдэг завсрын файлууд:

\keyword{main.aux} -- үүнийг \LaTeX{} үүсгэдэг.

\keyword{main.bbl} -- үүнийг BibTeX үүсгэдэг, устгасан бол \file{main.aux} --г ажиллуулахад BibTeX дахин үүсгэдэг. Танд байгаа бүх ишлэлийг \file{.bib} файл  агуулдаг ч \file{.bbl} --д зөвхөн тезист бодитоор ашигласан ишлэлийн мэдээлэл л хадгалагддаг.

\keyword{main.blg} -- үүнийг BibTeX үүсгэдэг.

\keyword{main.lof} -- үүнийг \LaTeX{} үүсгэдэг бөгөөд \emph{Зургийн жагсаалт} --ыг яаж үүсгэхийг хэлж өгдөг.

\keyword{main.log} -- үүнийг \LaTeX{} үүсгэдэг бөгөөд алдаа, сануулгын мэдээллийг агуулдаг.

\keyword{main.lot} -- үүнийг \LaTeX{} үүсгэдэг бөгөөд \emph{Хүснэгтийн жагсаалт} --ыг яаж үүсгэхийг хэлж өгдөг.

\keyword{main.out} -- үүнийг \LaTeX{} үүсгэдэг.

Энэ бүх урт жагсаалтаас \file{.bib}, \file{.cls} болон \file{.tex} өргөтгөлтэй файлууд хамгийн чухал бөгөөд бусдыг нь \LaTeX{}, BibTeX түр зуур үүсгэдэг.

%-------------------------------------------------------------------------------

\section{Өөрийн мэдээллийг \file{main.tex} файлд бөглөх}\label{FillingFile}

Өөрийн мэдээллийг бөглөх замаар тезисийн загварыг өөртөө тохируулах хэрэгтэй. Ингэхдээ өөрийн таалдаг \LaTeX{} орчин, эсвэл текст засварлагч ашиглан \file{main.tex} файлыг засварлана.

Файлыг нээгээд \emph{THESIS INFORMATION} блок руу очвол \emph{University Name}, \emph{Department Name}, г.м. зүйлийг харна.

Өөрийн болон сургуулийн тухай мэдээллээ бөглө. Мөн веб холбоосыг оруулах бол сайтын бүтэн URL хаягийг \code{http://} командад оруулж өгнө.

Засвар хийж, хадгалаад \file{main.tex} файлыг дахин хөрвүүлнэ. Таны оруулсан бүх мэдээлэл, холбоосны хамт PDF болно. Одоо тезисээ сайжруулах ажлыг эхэлж болно!

%-------------------------------------------------------------------------------

\section{\code{main.tex} файлын тайлбар}

Тезисийн бүтэц \file{main.tex} файлд байна. Энэ файлын дэлгэрэнгүй тайлбарууд \LaTeX{} кодоор үүссэн хуудас, агуулга, түүний хэвшүүлэлтийг тайлбарлана. Баримтын томоохон хэсгүүд томоор бичсэн гарчигтай блокоор тусгаарлагдсан болно. Эхэлж харахад маш их \LaTeX{} код харагдах боловч эдгээр нь бүхэлдээ хэвшүүлэлт бөгөөд түүнийг бодож хийсэн болохоор санаа зовох хэрэггүй. 

Эхлээд нүүр хуудсанд таны мэдээлэл зөв байгаа эсэхийг шалгаарай. Дараа нь тезис бичих төлөвлөгөө болон гүйцэтгэлтэй холбоотой хуудас байна. Гадаадын сургуулийн тезист энэ хэсэг байдаггүй ч манай улсад бакалаврын түвшинд шаарддаг. Тэгэхээр үүнийг өөрийнхөөрөө бөглөөрэй.

Тезисийн мэдэгдэлд танай сургууль бичсэнээс өөрийг хүсэж болно. Тэгвэл шаардлагатай мэдээллийг \emph{DECLARATION PAGE} блокт солиод бичихэд л болно.

Дараа нь гарч ирэх ишлэлийн хуудсанд өөрийн дээдэлдэг сургаал үгийн ишлэлийг оруулаарай. Энэ хуудас заавал байх зүйл биш.

Тезисийн хураангуй бол таны ажлын гол санаа, үр дүнг цөөн үгээр нэгтгэн илэрхийлэх ёстой.

Дараачийнх нь талархлын хуудас. Хэн нэгэнд, түүний дотор судалгааны багтаа зориулсан талархлын үгсийг энд бичнэ. Энэ хуудас байх эсэхийг та өөрөө шийдээрэй.

Агуулга, зураг, хүснэгтийн зэрэг хуудсыг та гараар бүтээх, засварлах хэрэггүй. Араас нь гарах товчилсон үгс, физик тогтмол болон таних тэмдгийн нэмэлт хуудас таны тезист (ялангуяа техникийн) байж болох юм. Ийм хуудсыг оруулснаар уншигч интернет болон лавлахаас хайхын оронд нэг дороос тодорхой тэмдэг, товчлолын утгыг авах бололцоотой болно.

Таних тэмдгийн жагсаал ром болон грек цагаан толгойгоор зааглагдсан бол товчилсон үгс үсгийн дарааллаар автоматаар жагссан. 

Энэ ажлыг хэн нэгэнд зориулсан бол түүнээ дараачийн хуудсанд илэрхийлнэ үү.

Эцэст нь тезист оруулах бүлгийн блок. Хэрэгтэй мөрийн өмнөх комментын тэмдгийг арилгаад холбогдох бүлгийг бичнэ.Бүлэг бүр бие даасан файл байх ба \emph{Chapters} гэсэн хавтаст \file{Chapter1}, \file{Chapter2}, г.м. нэртэй байна. Үүний адилаар хавсралтуудыг оруулна. Хавсралт бүр бие даасан файл болж \emph{Appendices} хавтаст байршина.

Удиртгал, бүлгүүд, хавсралтуудын араас эцэст нь ном зүй орно. Ном зүйг дугаарлах хэлбэрээр илэрхийлэхдээ онлайн бүтээлийн холбоосыг оролцуулаад бүх төрлийн ном зүйн материалыг тусгаж болно. Уншигч холбоосоор нь дамжаад эх материалд хандаж болно гэдэг ямар гайхалтай болохыг хэлүүлэх юун. Мэдээж үүний тулд URL мэдээллийг BibTex --д хэлж өгөх хэрэгтэй.

%-------------------------------------------------------------------------------

\section{Тезисийн боломж ба зөвшилцөл}\label{ThesisConventions}

Энэ загварын сайн зүйлийг авахын тулд та зарим нэг зөвшилцлийг дагах хэрэгтэй болно.

Хамгийн чухал (бас хамгийн хүнд) нь урт/том баримтын нэгдлийг хангах явдал. Тодорхой зөвшилцөл болон юмыг хийх арга зам (Хийх ажлын жагсаалт зэрэг) нь ажлыг хялбар болгодог. Мэдээж, энэ бүх нэмэлт зүйлийг та өөрөө зохицуулж чадна.

\subsection{Хэвлэх формат}

Энэ загвар тезисийг хуудасны нэг талд хэвлэхээр зохиогдсон. Хоёр талаар хэвлэх бол \file{main.tex} файлын \code{documentclass} командын \option{oneside} сонголтыг коммент болгох хэрэгтэй.

Хуудасны сонголтоос хамаарч хуудасны толгойд гарах мэдээлэл өөрчлөх ба хуудасны дугаар доод хэсэгт голлож байрлана.

Текстийг 12 --тийн фонтоор, мөр хоорондын зай нэг байхаар, тезисийг A4 хэмжээтэй цаасан дээр хэвлэхээр тохируулсан. Мөн хуудасны зах талбайн хэмжээ ШУТИС -ийн төгсөлтийн ажил бичих шаардлагатай нийцэж байгаа болно. Шаардлагатай бол өөрчлөлтийг \file{main.tex} эхэнд байгаа сонголтоос хийж болно. 

\subsection{Ишлэл}

Ном зүйг хэвшүүлэх,  \parencite{Reference1} маягийн ишлэл оруулахын тулд \code{biblatex} багцыг ашигласан. Олон ишлэлийг таслалаар тусгаарлах (\parencite{Reference2,Reference1} г.м.) ба гурваас илүү зохиогчтой (\parencite{Reference3}) бүтээлийг хэрхэн харуулахыг автоматаар шийднэ. Ишлэлийг яаж хийхийг \file{Chapter1.tex} файлын эх кодоос харна уу. 

Шинжлэх ухааны ишлэлийг цэг, таслал зэрэг тусгаарлагчийн \emph{өмнө} оруулдаг.  Хуудасны доор бичигдсэн\footnote{Үүн шиг ишлэл} --г мөн адилхан оруулна. Үүнийг өөрчилж болох ч тезисийн турш нэгдмэл зөвшилцлийг хангах нь чухал. Хөл ишлэл нь өөрөө бүтэн, тайлбар өгүүлбэр байх ёстой.

Ном зүйн бичлэг нь APA ишлэлийн загвартай төстэй. Үүнийг энэ баримтын төгсгөлөөс харж болно.

\subsection{Хүснэгт}

Хүснэгт бол үр дүнг харуулах чухал арга. Энэ кодын үүсгэсэн хүснэгтийг харна уу.

{\small
\begin{verbatim}
\begin{table}
\caption{Судалгааны 4 группын X болон Y эмчилгээний үр дүн}
\label{tab:treatments}
\centering
\begin{tabular}{l l l}
\toprule
\tabhead{Groups} & \tabhead{Эмчилгээ X} & \tabhead{Эмчилгээ Y} \\
\midrule
1 & 0.2 & 0.8\\
2 & 0.17 & 0.7\\
3 & 0.24 & 0.75\\
4 & 0.68 & 0.3\\
\bottomrule\\
\end{tabular}
\end{table}
\end{verbatim}
}

\begin{table}
\caption{Судалгааны 4 группын X болон Y эмчилгээний үр дүн}
\label{tab:treatments}
\footnotesize
\centering
\begin{tabular}{c c c}
\toprule
\tabhead{Групп} & \tabhead{Эмчилгээ X} & \tabhead{Эмчилгээ Y} \\
\midrule
1 & 0.2  & 0.8\\
2 & 0.17 & 0.7\\
3 & 0.24 & 0.75\\
4 & 0.68 & 0.3\\
\bottomrule\\
\end{tabular}
\end{table}

Хүснэгт рүү table орчинд тодорхойлсон шошгоор ишлэл хийж болно. Ишлэл, түүний шошгын жишээг \file{Chapter1.tex} файлаас харна уу (Хүснэгт~\ref{tab:treatments} г.м.).

\subsection{Зураг}

Таны тезист \emph{Figures} хавтасны олон зураг орох нь тодорхой. Дараах шиг загвар кодыг ашиглан тезисдээ зураг оруулж болно:
\begin{verbatim}
\begin{figure}
\centering
\includegraphics[scale=0.7]{Figures/Electron}
\caption[Электрон]{Электрон (зураачийн төсөөлөл).}
\label{fig:Electron}
\end{figure}
\end{verbatim}
Эх кодыг харна уу. Энэ кодыг эх файлд оруулснаар доорх электроны зургийг үүсгэнэ. Зургийн хэмжээг 70\% болгож багасгасныг анхаар!

\begin{figure}[ht]
\centering
\includegraphics[scale=0.7]{Figures/Electron}
\caption[Электрон]{Электрон (зураачийн төсөөлөл).}
\label{fig:Electron}
\end{figure}

Зураг тэр бүр таны бичсэн газарт ордоггүй. Байршил хуудас дээрх зурагт зориулсан талбайн хэмжээ ямар вэ гэдгээс хамаардаг. Заримдаа зургийг байх ёстой газар нь шууд оруулахад талбай дутдаг тул \LaTeX{} түүнийг дараачийн хуудасны эхэнд тавьдаг. Зургийн байршил бол \LaTeX{} --н ажил болохоор сайхан харагдуулахад л анхаарах хэрэгтэй!

Зургийг ишлэх (Зураг~\ref{fig:Electron} г.м.) тохиолдолд тэрээр тайлбартай байх ёстой. Команд \verb|\caption| нь дунд хаалтанд бичигдэж, \emph{Зургийн жагсаалтад} харагддаг зургийн нэр, их хаалтанд бичигддэг илүү тайлбар -- урт текст гэсэн хоёр хэсэгтэй байдаг. 

\subsection{Томьёо}

Таны тезис  математикийн хүндхэн агуулгатай бол \LaTeX{} түүнийг сайхан харуулж чадна.

\enquote{Not So Short Introduction to \LaTeX} (\href{http://www.ctan.org/tex-archive/info/lshort/english/lshort.pdf}{CTAN дээр байгаа}) математикийг бичих ихэнх тохиолдолд юу мэдэх хэрэгтэйг хэлж өгнө. Илүү их мэдээлэл хэрэгтэй бол AMS --аас гаргасан \enquote{A Short Math Guide to \LaTeX} нэртэй зөвлөмжийг \url{ftp://ftp.ams.org/pub/tex/doc/amsmath/short-math-guide.pdf} хаягаар татан авч болно.

Тогтоох хэрэгтэй \LaTeX{} --н олон тэмдэгт байдаг. Маш нийтлэг тэмдэгтийг \href{http://ctan.org/pkg/comprehensive}{The Comprehensive \LaTeX~Symbol List} --ээс олж болно.

Тэгшитгэлийг дараах байдлаар бичихэд \LaTeX{} автоматаар дугаар олгодог.
\begin{verbatim}
\begin{equation}
E = mc^{2}
\label{eqn:Einstein}
\end{equation}
\end{verbatim}

Энэ код А.Энштейний алдартай тэгшитгэлийг гаргана.
\begin{equation}
E = mc^{2}
\label{eqn:Einstein}
\end{equation}

Бичсэн бүх тэгшитгэлд (текстийн дунд ороогүй) \LaTeX{} автоматаар дугаар олгодог. Дугааргүй тэгшитгэл бичихийг хүсвэл дараах хэлбэрийг ашиглаарай.

\begin{verbatim}
\[ ax^{2}+bx+c=0 \]
\end{verbatim}
Дээрх код дараах томьёог гаргана:
\[ ax^{2}+bx+c=0 \]


%-------------------------------------------------------------------------------

\section{Сэдэв ба дэд сэдэв}

Тезисээ зөв хэмжээтэй сэдэв, дэд сэдэвт хуваах хэрэгтэй. \LaTeX{} нь эх файлд бичигдсэн \verb|\chapter{}|, \verb|\section{}|  ба \verb|\subsection{}| командаар гарчгийг автомаар үүсгэдэг.

Гарчигт сэдвийн 3 түвшинг жагсаадаг. Команд \verb|chapter{}| бол тэг (0) түвшин. Команд \verb|\section{}|, \verb|\subsection{}|, \verb|subsubsection{}| нь тус тус 1, 2, 3 --р түвшин болно. Ямар түвшин хүртэл гарчигт оруулахыг \file{MUSTThesis.cls} файлд заасан. Өөрчлөх хэрэгтэй бол \file{main.tex} файлд хийж болно..

%-------------------------------------------------------------------------------

\section{Төгсгөлд нь}

Та энэхүү мини зөвлөмжийн төгсгөлд ирлээ. Одоо та нэрийг нь өөрчилж, эсвэл дараад өөрийн \file{Chapter1.tex} файл болон тезисийн бусад хэсгийг бичээрэй. Тезисийн бүтэц, тогтолцоог тогтоох ажлыг таны өмнөөс хийлээ. Түүнийг бөглөх таны ажил л үлдлээ!

Амжилт хүсье!

\begin{flushright}
Зөвлөмжийг бичсэн ---\\
Sunil Patel: \href{http://www.sunilpatel.co.uk}{www.sunilpatel.co.uk}\\
Vel: \href{http://www.LaTeXTemplates.com}{LaTeXTemplates.com}
\end{flushright}
