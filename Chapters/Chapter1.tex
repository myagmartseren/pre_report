% Бүлэг 1

\chapter{Proxy Re-Encryption схемийн онолын хэсэг} % Бүлгийн нэр
\label{Chapter1} % Энэ бүлэг рүү ишлэл хийх бол \ref{Chapter1} командыг ашигла 
\pagecolor{white}
%-------------------------------------------------------------------------------

% Агуулгад ашигласан хэвшүүлэлтийн зарим командын тодорхойлолт
\newcommand{\keyword}[1]{\textbf{#1}}
\newcommand{\tabhead}[1]{\textbf{#1}}
\newcommand{\code}[1]{\texttt{#1}}
\newcommand{\file}[1]{\texttt{\bfseries#1}}
\newcommand{\option}[1]{\texttt{\itshape#1}}

%-------------------------------------------------------------------------------
%	SECTION 1
%-------------------------------------------------------------------------------
\section{Шифрлэлт, түүний ач холбогдол, ангилал, хэрэглээ}

Encryption is the process of converting plaintext or readable data into ciphertext or unreadable data using an encryption algorithm. The ciphertext can only be decrypted and read by authorized parties who possess the decryption key. Encryption is a critical tool for protecting sensitive information, ensuring the privacy of individuals and organizations, and securing digital communications. Here are some of the significance, importance, and applications of encryption:

Confidentiality
Encryption helps to maintain the confidentiality of sensitive data by making it unreadable to unauthorized parties. It ensures that only authorized parties can access and read the information, protecting it from theft, eavesdropping, or interception.

Privacy
Encryption ensures the privacy of individuals and organizations by securing their personal and sensitive information. It allows individuals to control who can access their information and how it can be used, reducing the risk of identity theft, fraud, or other forms of privacy violations.

Authentication
Encryption helps to ensure the authenticity of data and messages by verifying the identity of the sender and ensuring that the message has not been tampered with during transmission. This is particularly important in online transactions, where the authenticity of data and messages is crucial to prevent fraud and ensure trust.

Data Integrity
Encryption helps to maintain the integrity of data by ensuring that it has not been tampered with or altered during transmission or storage. It allows data to be stored and transmitted securely without the risk of unauthorized modifications, ensuring the accuracy and reliability of information.

Applications
Encryption is used in a wide range of applications, including secure online transactions, digital signatures, secure email communication, online banking, e-commerce, and data storage. It is also used to secure sensitive information in industries such as healthcare, finance, and government, where privacy and confidentiality are paramount.

In conclusion, encryption is a critical tool for protecting sensitive information, ensuring privacy, and securing digital communications. Its significance and importance continue to grow as digital technologies become more pervasive and the threats to digital security become more sophisticated. Encryption technology is continually evolving to meet the increasing security needs of individuals and organizations, ensuring that sensitive information remains secure and confidential.

%-------------------------------------------------------------------------------
%	SECTION 2
%-------------------------------------------------------------------------------
\section{Орчин үеийн ширфлэлтийн схемүүд}
Homomorphic encryption: While homomorphic encryption allows computations to be performed on encrypted data, it does not provide delegation or access control features like PRE.

Secure multi-party computation: SMPC allows multiple parties to jointly compute a function over their private inputs without revealing those inputs to each other, but it does not provide delegation or access control features like PRE.

Attribute-based encryption: ABE allows access to data to be controlled based on certain attributes, but it does not provide delegation or re-encryption features like PRE.
%-------------------------------------------------------------------------------
%	SECTION 3
%-------------------------------------------------------------------------------

\section{Proxy Re-Encryption схем}

Прокси дахин шифрлэлт нь 
Прокси дахин шифрлэлт нь нийтийн түлхүүрээр шифрлэлтийн нэг хэлбэр бөгөөд хэрэглэгч Алиса-ийн шифрийг Bob-д тайлах боломжийг олгодог.

Үндсэн хоёр төрөлтэй.

Unidirectional PRE: Зөвхөн нэг талдаа дахин шифрлэх боломжтой.

Bidirectional PRE: 2 талдаа дахин шифрлэх боломжтой.
\\

Some features of PRE schemes include:

Delegation: PRE allows data owners to delegate access to their data to third-party entities, without giving them complete access to the data.

Access control: PRE allows data owners to control who can access their data and under what circumstances, even after the data has been shared.

Efficiency: PRE can be more efficient than traditional re-encryption techniques, as it does not require the data to be decrypted and re-encrypted.

Security: PRE provides a high level of security, as the proxy does not have access to the data itself and can only transform the encrypted dat

%-------------------------------------------------------------------------------
%	SECTION 3
%-------------------------------------------------------------------------------

\section{Бүлгийн Дүгнэлт}