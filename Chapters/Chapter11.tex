% БҮЛЭГ 1


\chapter{Судалгаа} % Бүлгийн нэр
\label{Chapter1} % Энэ бүлэг рүү ишлэл хийх бол \ref{Chapter1} командыг ашигла 

%-------------------------------------------------------------------------------

% Агуулгад ашигласан хэвшүүлэлтийн зарим командын тодорхойлолт
\newcommand{\keyword}[1]{\textbf{#1}}
\newcommand{\tabhead}[1]{\textbf{#1}}
\newcommand{\code}[1]{\texttt{#1}}
\newcommand{\file}[1]{\texttt{\bfseries#1}}
\newcommand{\option}[1]{\texttt{\itshape#1}}

%-------------------------------------------------------------------------------
\pagecolor{white}
\section{Монгол дахь автомашины зогсоолын судалгаа}

Өнөө үед автомашингүйгээр дэлхийн аль ч улс оронд амьдрахад хэцүү болсон. Улаанбаатар хотод 2019 оны 3 сарын байдлаар 501934 автомашин тоологдож байсан бөгөөд автомашин ихсэхийн хэрээр түүнийг байрлуулах зогсоол нь  тулгамдсан асуудлуудын нэг болоод байгаа юм.  Авто зогсоолыг: 
\begin{enumerate}
	\item \textbf{Авто тээврийн хэрэгслийн зогсоол} (цаашид авто зогсоол гэх) - зөвхөн авто тээврийн хэрэгсэл байрлуулах (зогсоолын) зориулалттай барилга, байгууламж (барилга, байгууламжийн хэсэг), түр буюу удаан хугацаагаар байрлуулах зориулалтын талбай.
	\item \textbf{Газар дээрх битүү авто зогсоол} - гадна хашлага бүтэц бүхий авто тээврийн хэрэгслийн зогсоол.
	\item \textbf{Задгай авто зогсоол} - гадна хашлага бүтэцгүй авто тээврийн хэрэгслийн зогсоол. Задгай авто зогсоолд мөн харьцангуй урт хэмжээний хоёр талаасаа задгай байгууламж хамаарна. Тухайн талын хашлага бүтцийн гадаргуугийн нийт талбайн 50 хувиас багагүйг нүх эзэлж байвал задгай авто зогсоолд хамааруулна.
	\item \textbf{Налуу пандус (рамп)-тай авто зогсоол} – тойруугаар тогтмол өгсөх (буух) эсвэл, давхар хооронд холбох пандус буюу налуу замаар авто тээврийн хэрэгсэл түвшин хооронд шилжих боломж бүхий авто зогсоол.
	\item \textbf{Механикжсан авто зогсоол} – жолоочийн оролцоогүйгээр зориулалтын механикжсан төхөөрөмжөөр авто тээврийн хэрэгслийг байрлуулах зогсоолд зөөвөрлөн байрлуулах боломжтой зогсоол.
\end{enumerate}
гэж тус тус нэрлэдэг байна.\cite{ners} \\

Авто зогсоол нь төлбөртэй, төлбөргүй гэсэн хоёр янз байгаагаас төлбөртэй нь илүү аюулгүй, амар байдаг гэж үздэг. Харин төлбөртэй зогсоолын төлбөрийг тусгай ажилтан хураан авдаг нь дутагдалтай байгаа учир зогсоолуудыг автоматжуулах хэрэгтэй болоод байна. Зогсоолуудыг автоматжуулснаар тухайн байгууллага болон зогсоолоор үйлчлүүлэгч нарт тухайн зогсоолд автомашины зай хэр байгааг зогсоолд орохоос өмнө мэдэх, цаг хэмнэх, мөнгөө хэмнэх гэх мэт олон давуу талыг олгоно.\\
 
Улаанбаатар хотод үйл ажиллагаа явуулж буй томоохон байгууллагуудын зогсоолыг ажиглахад: 
\begin{itemize}
	\item Үйлчлүүлсэн ч зогсоолын төлбөр төлдөг
	\item Үйлчлүүлсэн тохиолдолд зогсоолын төлбөр төлдөггүй, үйлчлүүлээгүй тохиолдолд зогсоолын төлбөр төлдөг гэсэн хоёр янз байна.
\end{itemize}

\begin{figure}[!ht]
	\centering
	\includegraphics[scale=1]{1-1}
	\caption{Төлбөртэй зогсоолын гарц}
	\label{fig:1-1}
\end{figure}
%-------------------------------------------------------------------------------
\subsection{Монгол дахь ухаалаг авто зогсоолын жишээ }
Манай улсын авто зогсоолын асуудлыг шийдэхэд тохирсон хамгийн том жишээ бол ''E-MART'' худалдааны төвийн ухаалаг зогсоол юм. Тус зогсоол нь барилгын 3,4-р давхрын зогсоол руу орохдоо дотор нь хэчнээн сул зогсоол байгааг гадаах дижитал самбараас харах бөгөөд ороод ногоон гэрэл ассан зогсоолд машинаа байрлуулдаг.
\begin{figure}[!ht]
	\centering
	\includegraphics[scale=0.65]{1-2}
	\caption{E-MART худалдааны төвийн ухаалаг зогсоолын гарц}
	\label{fig:1-2}
\end{figure}
\section{Япон дахь автомашины зогсоолын судалгаа }
Японд хэдийгээр машин хямдхан ч гэлээ автомашинтай байна гэдэг хэцүү хэрэг бөгөөд авто зогсоолын бэрхшээл зайлшгүй тулгарна. Хамгийн жижигхэн хотод ч авто зогсоол нь төлбөртэй байдаг. Гэвч Японы гудамжинд ил зогссон машин бараг байхгүй байдаг нь бүрэн автоматжсан, өргөдөг шаттай, машиныг хэрэгтэй давхарт гаргаж өгөх, байрлуулах, буцааж буулгаж ирэх зориулалт бүхий роботоор тоноглогдсон авто байшин их барьдаг бөгөөд бүх худалдааны төв, оффист ийм систем үйлчилж үйл явцыг оператор хянана. Ихэнх томоохон үйлчилгээний газрууд үнэгүй зогсоолтой байдаг. Мөн хамгаалагч зохицуулагч бүхий хамгийн энгийн зогсоол ч бас байдаг ч ихэнх авто зогсоолууд автоматжуулагдсан байдаг.\\

\begin{figure}[!ht]
	\centering
	\includegraphics[scale=0.3]{1-3}
	\caption{Автоматжуулагдсан автомашины зогсоол}
	\label{fig:1-3}
\end{figure}

Автомашинаа зогсоолд байрлуулаад машинаа аваад гарахдаа пос машинд өөрөө төлбөрөө төлөн гардаг. Мөн дугаар бүхий зогсоолуудад автомашин байрлуулахад автомашин доороос түгжигч гарч ирж түгждэг. Машин зогсоолд орсны дараа хөдлөх боломжгүй болох ба төлбөрөө төлөөгүй бол гарахгүй. Хэрэглэгч автомашинаа авахад өөрийн машины зогсоолын дугаарыг пос машинд төлбөрийг төлөөд авах боломжтой (\ref{fig:1-3} дугаар зураг). Зарим газар хаалт байхгүй ч төлбөрөө өөрөө төлөөд явдаг. 


\subsection{Японы хурдны зам}
 
Японы хурдны зам дээр ЕТС (Electronic toll collection) ашигладаг бөгөөд зай нь 10м орчим байдаг.(\ref{fig:1-4} дүгээр зураг). Хамгийн урд ЕТС-н хүлээн авах төхөөрөмж байрлах бөгөөд автомашин дотор ЕТС-н карт уншигч байрлана. ЕТС-н хүлээн авагчид карт уншигчаас картын мэдээлэл очно. Карт уншсан нөхцөлд нэвтрэх эрх нээгдэнэ. Хэрвээ карт байхгүй бол карт хүлээн авагчийн дэргэд байрлах бэлэн мөнгөний машинд төлбөрийг төлнө. 

\begin{figure}[!ht]
	\centering
	\includegraphics[scale=1]{1-4}
	\caption{Японы хурдны зам дээр ЕТС}
	\label{fig:1-4}
\end{figure}

\section{Техникийн үнэ}

\begin{tabular}[b]{|c|p{6 cm}|c|c|c|c|}
	\hline
№  & Нэрс & нэгж & Тоо & нэгж үнэ & Нийт үнэ \\ \hline
1  & PLC SIEMENS S7-224 XP CPU AC/DC/RLY & ком & 1 & 450000 & 450000 \\ \hline
2  & LED MODULE  P10  32x16 dot  & ш & 1 & 40000 & 40000 \\ \hline
3  & BUTTON BOX-4 & ком & 1 & 10000 & 10000 \\ \hline
4  & Power Supply 5V DC/3A 12V DC/2A & ком & 1 & 45000 & 45000 \\ \hline
5  & RS232/485  Isolated HUB 4 port RS485 (ZLAN-9440) & ком & 1 & 150000 & 150000 \\ \hline
6  & Siemens  RS-232/PPI Multi-Master cable & ш & 1 & 30000 & 3000 \\ \hline
7  & USB/RS232 ZTEK converter & ш & 1 & 18000 & 18000 \\ \hline
8  & AT89C51ED2   Микроконитроллер 22.1184 MHz & ком & 1 & 15000 & 15000 \\ \hline
9  & 74HC245 Buffer & ш & 1 & 1000 & 1000 \\ \hline
10  & 74HC165  Shift Register & ш & 1 & 1000 & 1000 \\ \hline
11  & MAX232, MAX485 & ш & 2 & 800 & 1600 \\ \hline
12  & Тэжээл : 7805-  12V/5V,  1N4005 & ком & 1 & 1500 & 1500 \\ \hline
13  & Конденсатор:  10mkF/ 25V,1mkF/50V  ,33mkF/25V & ш & 10 & 300 & 3000 \\ \hline
14  & Эсэргүүцэл : 10kOm, 100Om & ш & 1 & 20 & 20 \\ \hline
15  & Микросхемийн суурь:44p, 20p, 16p, 8p & ш & 4 & 700 & 2800 \\ \hline
16  & PCB хавтан  & ш & 1 & 9000 & 9000 \\ \hline
16  & Connector 3x1, IDE 2x8,  DB9/M & ш & 5 & 500 & 2500 \\ \hline
17  & IDE Flat cable, Power and signal Cable,PLC Com 0,1 Freeport DB9 RS485 Serail cable Монтаж утас, гагнуурын тугалга гэх мэт & ком & 1 & 15000 & 15000 \\ \hline
  &          Нийт  (төгрөг)  &  &  &  & 795420 \\ \hline
\end{tabular}
