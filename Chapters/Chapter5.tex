\chapter{Томъёо хэрхэн бичих вэ?} % Зарим нэг зөвлөмж

\label{Chapter5} % Энэ бүлэг рүү ишлэл хийх бол \ref{Chapter2} командыг ашигла 
\pagecolor{white}
%-------------------------------------------------------------------------------
%	SECTION 1
%-------------------------------------------------------------------------------

\section{Математик горим}

Доорх \ref{eq1} дүгээр томъёонд харуулсан тэгшитгэлээр ......\cite{zorigt1}
 

\begin{equation}
\int_{s}^{}rot E \quad dS=-\int_{s}\frac{\partial B}{\partial t}dS  \qquad \textit{Бодлого1}
\label{eq1}
\end{equation}

\[ 
\begin{bmatrix}
V_{e1} \\
V_{e2} \\
V_{e3} 
\end{bmatrix}=
\begin{bmatrix}
1 & x_1 & y_1 \\
1 & x_2 & y_2 \\
1 & x_3 & y_3
\end{bmatrix}
\cdot \begin{bmatrix}
a \\
b \\
c
\end{bmatrix}\]
\[
V(x,y)=\sum_{i=1}^{3}\alpha_i(x,y)V_{ei}
\]


Дохионы давтамж $\omega=4000 rad/sec$ , чадал нь $p=30\mu W $ байсан бол.... \cite{uguulel}


\section{Хүснэгт}

\begin{table}[!ht]
	\centering
	\caption{Жишээ хүснэгт}
	\label{table1}
	
	\begin{tabular}{|p{2cm}|p{2cm}|p{2cm}|p{7.5cm}|}
	\hline
		\multirow{2}{*}{ нэр} & 1.00 & 3.5& “Электроникийн үндэс” хичээлийг судaлж буй оюутнуудыг заавар, аргачлалын дагуу туршилт, хэмжилтийн ажил гүйцэтгэх, тэмдэглэл хийхэд зориулсан сургалтын материал болно \cite{online1}. \\
	%\hline	
	  & 2.00 & 4.5& “Электроникийн үндэс” хичээлийг судaлж буй оюутнуудыг заавар, аргачлалын дагуу туршилт, хэмжилтийн ажил гүйцэтгэх, тэмдэглэл хийхэд зориулсан сургалтын материал болно \cite{vhdl}. \\
	\hline
\end{tabular}	

\end{table}


	
