% Бүлэг 3

\chapter{Төсөл} % Зарим нэг зөвлөмж

\label{Chapter3} % Энэ бүлэг рүү ишлэл хийх бол \ref{Chapter2} командыг ашигла 


%-------------------------------------------------------------------------------
%	SECTION 3
%-------------------------------------------------------------------------------
\pagecolor{white}
\section{Ерөнхий бүтэц} 
Авто зогсоолын төлбөрийн ухаалаг систем нь PLC контроллер, LED матрицан дэлгэц, гэрлэн дохио, мэдрүүр, автомат хаалт, удирдлага-хяналтын компьютерээс бүрдэнэ.\\
\begin{figure}[!ht]
	\centering
	\includegraphics[scale=0.6]{3-1}
	\caption{Авто зогсоолын төлбөрийн системийн бүтэц}
	\label{fig:3-1}
\end{figure}
Мэдрүүрийн харалдаа авто машин орж ирж зогссоноор PLC контроллер ямар үйлдэл хийхээ сонгоно. Орох хэсгийн мэдрүүр мэдэрсэн тохиолдолд гэрлэн дохио улаан болж хаалт нээгдэнэ. Гарах хэсгийн мэдрүүр мэдэрсэн тохиолдолд LED дэлгэцэд үнийн мэдээлэл гарч хаалт эхэлнэ. Аваарын товчлуур дарагдсан тохиолдолд орох гарах хэсгийн мэдрүүрүүд ажиллагаагүй болж гэрлэн дохио шараар анивчиж хаалт нээгдэнэ. Энэ бүх үйлдлийг удирдлага-хяналтын компьютероос хянана. 
\section{ЛED дэлгэцийн удирдлага}
LED дэлгэцийг жолоочид зориулсан төлбөрийн мэдээлэл гаргахад хэрэглэж байгаа бөгөөд P10 загварын дэлгэцийн модулийг ашигласан. \\

Энэхүү модулийн удирдлагаар At89c51ed2 микроконтроллерийг сонгосон бөгөөд удирдлагын хавтангийн хэлхээний зарчмын схемийг боловсруулж, хавтанг угсарч ажиллуулсан юм. \\

\ref{fig:3-2} дугаар зурагт P10 модулийн хавтангийн хэлхээний зарчмын схемийг, \ref{fig:3-3} дугаар зурагт уг хавтанг удирдах контроллерийн хэлхээний зарчмын схемийг тус харууллаа. \\
\newpage
\begin{figure}[!ht]
	\centering
	\begin{turn}{90}
	\includegraphics[scale=0.4]{3-6}
	\end{turn}
	\caption{LED дэлгэцийн зарчмын схем}
	\label{fig:3-2}
\end{figure}
\newpage

\begin{figure}[!ht]
	\centering
	\begin{turn}{90}
	\includegraphics[scale=0.5]{3-5}
\end{turn}
	\caption{LED дэлгэцийн удирдлагын контроллерийн зарчмын схем}
	\label{fig:3-3}
\end{figure}

\newpage

At89c51ed2 микроконтроллер нь текст горимд ажиллах бөгөөд тогтмол санах ойд дэлгэцэнд дүрслэх тэмдэгтийн дүрсийн мэдээллийг урьдчилан хадгалсан байна. \\

Мөн дэлгэцийн синхрон удирдлагыг зохион байгуулахад хялбар байлгахын тулд дэлгэцийн санах ойг үүсгэсэн бөгөөд энэ санах ойд дэлгэцэнд гэрэлтэх цэгийн мэдээллийг хадгална. Дэлгэцийн санах ой 0,1 гэж дугаарлагдсан хоёр буфферээс тогтох бөгөөд дан модультай дэлгэцийн хувьд зөвхөн 1-р санах ойн мэдээллийг уншиж дүрсэлж байгаа. Харин дэлгэц нэгээс олон модулиас тогтох үед хоёр буфферийг хоюуланг нь ашиглана. 
 
\begin{figure}[!ht]
	\centering
	\includegraphics[scale=0.8]{3-7}
	\caption{LED дэлгэцийн санах ойн хаяглалт}
	\label{fig:3-4a}
\end{figure}

Энэ контроллерийн програм нь үндсэн гурван хэсэгтэй бөгөөд үүнд: 
\begin{itemize}
	\item P10 дэлгэцийн мөрийн хаяглалтын A, B, цэгийн гэрэлтэлтийн R, синхрон удирдлагын CLK, SCLK, OE сигналуудыг өгөгдлийн портоороо хэлбэржүүлнэ. Үүнд цэгийн гэрэлтэлтийг дэлгэцийн санах ойд байгаа утгаар тодорхойлно. 
	\item Асинхрон цуваа сувгаар дэлгэцэнд гаргах мэдээллийг текст хэлбэрээр хүлээж авна. 
	\item Хүлээж авсан мэдээллийн тэмдэгт бүрд харгалзах дүрсийн мэдээллийг тогтмол санах ойгоос уншиж, дэлгэцийн санах ойд хадгална.
\end{itemize}	
	 Програмын алгоритмыг \ref{fig:3-4} дүгээр зурагт харуулж, Assembler хэлээр бичсэн програмын бүрэн кодыг хавсралтанд хавсаргав. 
\begin{figure}[!ht]
	\centering
	\includegraphics[scale=0.6]{3-2}
	\caption{LED дэлгэцийн удирдлагын алгоритм}
	\label{fig:3-4}
\end{figure}
\newpage

\subsection{ЛЕД дэлгэцийг удирдах туршилтын үр дүн}
Төслийн явцад хамгийн түрүүнд ЛЕД дэлгэцийн удирдлагын хэлхээг зохион бүтээж, туршин шалгасан бөгөөд \ref{fig:3-5} дугаар зурагт хамгийн анх хийсэн дэлгэцийн синхрон удирдлагын сигналууд зөв хэлбэржиж байгаа эсэхийг тодорхой координатад цэгэн гэрэл асааж шалгасан туршилтын үр дүнг харууллаа. 
\begin{figure}[!ht]
	\centering
	\begin{turn}{90}
	\includegraphics[scale=0.2]{3-8}
	\end{turn}
	\caption{LED дэлгэцийн синхрон удирдлагын сигналын туршилт}
	\label{fig:3-5}
\end{figure}

Харин \ref{fig:3-6} дугаар зурагт микроконтроллерийн тогтмол санах ойд хадгалсан тэмдэгтийн дүрсэн мэдээллийг зөв уншиж, дүрсэлж байгаа эсэхийг туршсан туршилтыг харуулав. Энэ туршилтыг гараас оруулсан тэмдэгтүүдийг цуваа сувгаар микроконтроллерт дамжуулж хийсэн. 

 \begin{figure}[!ht]
 	\centering
 	\includegraphics[scale=0.2]{3-9}
 	\caption{LED дэлгэцийн удирдлагын програмын туршилтын үр дүн}
 	\label{fig:3-6}
 \end{figure}

\section{PLC удирдлага}
Орох, гарах хаалганд авто машин дөхөж ирсэн эсэхийг мэдрэх мэдрүүрийн төлвийг унших, аваарийн товчны төлвийн унших, хаалтыг нээж, хаах, гэрэл дохиог удирдахад програмчлагддаг логик контроллер(PLC)-ийг хэрэглэсэн бөгөөд түүний програмын ерөнхий алгоритмыг \ref{fig:3-7} дугаар зурагт харуулж, ladder logic-ийн бүрэн кодыг хавсралтанд орууллаа. \\
 \begin{figure}[!ht]
	\centering
	\includegraphics[scale=0.8]{3-4}
	\caption{PLC програмын алгоритм}
	\label{fig:3-7}
\end{figure}

PLC контроллерийн програм дараах зарчимтай ажиллана. Үүнд: 
\begin{enumerate}
\item Удирдлага-хяналтын компьютерт өөрийн бүх төлвийг илгээж байх бөгөөд компьютерээс хаалтыг нээх хаах, аваарийн үеийн удирдлагыг авна.
\item Аваарийн үед аваарийн товч дарагдсан эсвэл компьютерээс аваарь үүссэн гэсэн удирдлагын мэдээлэл ирсэн үед гэрэл дохио шар өнгөөр анивчин асаж, орох гарах гарцын бүр хаалтууд нээгдэнэ. 
\item Орох гарцанд автомашин ирэхэд мэдрүүр мэдэрч орох хэсгийн гэрлэн дохио улаан болж бүртгэлийн бар код авахыг хүлээнэ. Бар код авснаар бүртгэл хийгдэж хаалтыг нээж, хугацааны тоолуур тоолж эхэлнэ. 
\item Гарах гарцын мэдрүүрт авто машин мэдрэгдсэн тохиолдолд LED дэлгэцэд тухайн авто машины төлбөрийн мэдээлэл харагдана. Төлбөр төлөгдсөн тухай мэдээлэл удирдлага-хяналтын компьютерээс ирэхэд гарах хэсгийн хаалтыг нээнэ. 
\item ЛЕД дэлгэц дээр дүрслэх текстэн мэдээллийг компьютерээс хүлээж авч, дэлгэцийн удирдлагын контроллерт дамжуулна.  
\end{enumerate}

\section{Авто зогсоолын системийн ерөнхий удирдлага}
Зогсоолын төлбөр тооцоог хийх, орох, гарах гарцын төлвийн мэдээллийг PLC контроллероос унших, шаардлагатай үед хаалт нээх, хаах, аваарийн үеийн удирдлага болон ЛЕД дэлгэцэнд гаргах төлбөрийн мэдээллийг илгээх зэрэг үйл ажиллагаатай удирдлага-хяналтын програмыг би LabView програм хангамжийг ашиглан гүйцэтгэсэн билээ. 

Уг удирдлага-хяналтын програмын харагдах байдлыг \ref{fig:3-8} дүгээр зургаар, ерөнхий алгоритмыг \ref{fig:3-9} дугаар зургаар тус тус харуулав. 
 \begin{figure}[!ht]
	\centering
	\includegraphics[scale=0.45]{A1}
	\caption{Удирдлага-хяналтын LabView програмын харагдах байдал}
	\label{fig:3-8}
\end{figure}

Удирдлага хяналтын компьютерийн дэлгэцийн зүүн талд орох гарах гарцыг дүрсэлсэн бөгөөд энд машин ирсэн эсэхийг илтгэх дохио болон хаалтны байрлалыг харуулна.

Дэлгэцийн баруун талд PLC контроллерийн портуудыг төлвийг хянах, ЛЕД дэлгэцэн дээр харагдаж байгаа мэдээллийг хүнд харуулах хоёр жижиг дэлгэц байна.
 \begin{figure}[!ht]
	\centering
	\includegraphics[scale=0.6]{3-3}
	\caption{Удирдлага-хяналтын програмын алгоритм}
	\label{fig:3-9}
\end{figure}
LabView програмын бүрэн блок схемийг хавсралтанд орууллаа. 

\newpage
\section{Авто зогсоолын төлбөрийн ухаалаг системийн туршилтын үр дүн}
Авто зогсоолын ухаалаг системийн удирдлагын нэгдсэн туршилтуудыг 
\begin{itemize}
	\item аваарийн товч дарагдсан үеийн удирдлага (зураг \ref{fig:3-10})
	\item зогсоол руу орох үеийн үеийн удирдлага (зураг \ref{fig:3-11})
	\item зогсоолоос гарах үеийн удирдлага (зураг \ref{fig:3-12})
\end{itemize}
гэсэн гурван нөхцөлд хийж үр дүнг зургаар харууллаа. 
 \begin{figure}[!ht]
	\centering
\begin{subfigure}[b]{0.45\textwidth}
	\includegraphics[width=\textwidth]{3-10}
	\caption{PLC удирдлага}
\end{subfigure}
~
\begin{subfigure}[b]{0.45\textwidth}
	\includegraphics[width=\textwidth]{3-11}
	\caption{Хяналтын дэлгэц}
\end{subfigure}
	\caption{Аваарийн товч дарагдсан туршилт}
	\label{fig:3-10}
\end{figure}

Аваарийн товч дарагдахад гэрлэн дохиог шар өнгөөр анивчуулах ба орох, гарах хаалтыг нээнэ.\\
 \begin{figure}[!ht]
	\centering
	\includegraphics[scale=0.2]{3-12}
	\caption{Зогсоол руу орох үеийн туршилт}
	\label{fig:3-11}
\end{figure}

Автомашин зогсоол руу орох үед мэдрүүр мэдэрсний дараа орох хэсгийн гэрлэн дохио улаан болж бүртгэлийн бар код авахыг хүлээнэ. Бар код авснаар бүртгэл хийгдэж хаалт нээгдэн тоолуур тоолж эхэлнэ. \\
 \begin{figure}[!ht]
	\centering
	\begin{subfigure}[b]{0.60\textwidth}
		\includegraphics[width=\textwidth]{3-13}
		\caption{LED дэлгэц}
	\end{subfigure}
	~
	\begin{subfigure}[b]{0.33\textwidth}
		\includegraphics[width=\textwidth]{3-14}
		\caption{PLC удирдлага}
	\end{subfigure}
	\caption{Зогсоолоос гарах үеийн туршилт}
	\label{fig:3-12}
\end{figure}

Гарах хэсгийн мэдрүүрт авто машин мэдрэгдсэн тохиолдолд LED дэлгэцэд тухайн авто машины төлбөрийн мэдээлэл харагдана. Төлбөр төлөгдсөн тохиолдолд гарах хэсгийн хаалт нээгдэнэ. 

