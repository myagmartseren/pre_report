% Бүлэг 2

\chapter{Прокси дахин шифрлэлтэд суурилсан файл хуваалцах систем} % Зарим нэг зөвлөмж
\label{Chapter2} % Энэ бүлэг рүү ишлэл хийх бол \ref{Chapter2} командыг ашигла 
\pagecolor{white}

%-------------------------------------------------------------------------------
%	SECTION 1
%-------------------------------------------------------------------------------
\section{Прокси дахин шифрлэлт}

Прокси дахин шифрлэлт нь нийтийн түлхүүрээр шифрлсэн өгөгдөлийг дахин ширфлэж өөр хувийн түлхүүрээр тайлах боломжийг олгодог.

Үндсэн хоёр төрөлтэй.
\begin{itemize}
    \item Нэг чиглэлт (Unidirectional PRE)
    \item Хоёр чиглэлт (Bidirectional PRE)
\end{itemize}

Нэг чиглэлт PRE (KE, RG, E, R, D) хэсгүүдээс тогтоно.

\begin{enumerate}
    \item Алис, Боб болон Чарли хувийн болон нийтийн түлхүүрийг үүсгэнэ. (KE)
    \item Алис Боб-д зориулж өгөгдлөө шифрлэж серверт байршуулна.
    \item Боб Алис-ын өгөгдлийг Чарли-тай хуваацлахын тулд RE(pkB,skB,pkC,skC∗) шифрлэж серверт байршуулна. Чарлигийн хувийн заавал шаардахгүй үүсгэж болно.
    \item Боб RE-г ашиглаж үүсэгсэн түлхүүрийг серверт явуулж Алисын файлыг дахин шифрлэж Чарли тайлах боломжтой болно.
\end{enumerate}

\begin{figure}[ht]
\centering
\includegraphics[scale=0.5]{Figures/pre}
\caption[Proxy Re-encryption scheme]{Proxy Re-encryption scheme}
\label{fig:PRE_Scheme}
\end{figure}

Давуу талууд:
\begin{itemize}
    \item Нууцлалыг сайжруулна: PRE нь оролцогч талуудын хувийн мэдээллийг задруулахгүйгээр өгөгдлийг хуваалцахыг зөвшөөрснөөр нууцлалыг сайжруулахад тусална. Энэ нь талууд нууцаар эсвэл хувийн нууц мэдээллийг задруулахгүйгээр мэдээллээ хуваалцахыг хүссэн тохиолдолд хэрэг болно.
    \item Нарийн төвөгтэй байдлыг багасгасан: PRE нь итгэмжлэгдсэн гуравдагч этгээдэд шифрлэлт болон шифрийг тайлах үйл явцыг удирдах боломжийг олгосноор шифрлэлт болон түлхүүрийн удирдлагын нарийн төвөгтэй байдлыг багасгахад тусална. Энэ нь ялангуяа олон талын оролцоотой, гол менежмент нь төвөгтэй, удирдахад хэцүү болж болзошгүй тохиолдолд хэрэг болно.
\end{itemize}

Сул талууд:
\begin{itemize}
    \item Проксид итгэх: PRE нь дахин шифрлэлтийг гүйцэтгхэд гуравдагч талын прокси дээр тулгуурладаг ба схемийн аюулгүй байдал нь прокси талаас хамаарна.
    \item Хязгаарлагдмал өргөтгөх чадвар: PRE нь өргөтгөх чадварын хувьд хязгаарлагдмал байж болно. Учир нь хэрэглэчдийн тоо нэмэгдэхийн хэрээр олон талыг дэмжихэд шаардлагатай дахин шифрлэлтийн түлхүүрүүдийн тоо хурдацтай өсөх болно. Энэ нь гол менежментийг төвөгтэй болгож, удирдахад хэцүү болгодог.
    \item Potential for replay attacks: PRE нь халдагч хариуг зогсоож хандах эрхийг өөрт ашигтай солих боломжтой. 
    \item Хүчингүй болгоход хүндрэлтэй байдал: PRE дахь өгөгдөлд хандах эрхийг цуцлах нь ялангуяа олон тал оролцсон тохиолдолд хэцүү байж болно. Хэрэв аль нэг талын дахин шифрлэлтийн түлхүүр алдагдсан бол бусад талуудын мэдээлэлд хандах эрхэд нөлөөлөхгүйгээр өгөгдөлд хандах эрхийг цуцлах нь хэцүү байж болно.
    \item Хязгаарлагдмал хэрэглээ: PRE нь харьцангуй шинэ бөгөөд шинээр гарч ирж буй технологи хэвээр байгаа бөгөөд илүү уламжлалт шифрлэлтийн схемүүдтэй харьцуулахад хэрэглээ нь хязгаарлагдмал байдаг. Энэ нь технологийг хэрэгжүүлэх, удирдах туршлагатай мэргэжилтнүүд бага байдаг.
\end{itemize}

%-------------------------------------------------------------------------------
%	SECTION 2
%-------------------------------------------------------------------------------
% \section{Хөгжүүлэх технологи, хэл сонгох}
%-------------------------------------------------------------------------------
%	SECTION 3
%-------------------------------------------------------------------------------
% \section{Хөгжүүлэлтийн орчин бэлдэх}